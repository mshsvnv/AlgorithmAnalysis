\chapter{Аналитическая часть}
В данной части работы будет описан алгоритм сортировки слиянием, а также рассмотрено использование многопоточности при его реализации.

\section{Описание алгоритма}
Алгоритм сортировки слиянием применяет принцип <<разделяй и властвуй>> для упорядочивания элементов в массиве.

Алгоритм сортировки состоит из нескольких этапов.
\begin{enumerate}
	\item \textit{Разделение}: исходный массив разделяется на две равные (или почти равные) половины. Это делается путем нахождения середины массива и создания двух новых массивов, в которые будут скопированы элементы из левой и правой половин.
	
	\item\textit{ Рекурсивная сортировк}а: каждая половина массива рекурсивно сортируется с помощью алгоритма сортировки слиянием. Этот шаг повторяется до тех пор, пока размер каждой половины не станет равным 1.
	
	\item \textit{Слияние}: отсортированные половины массива объединяются в один отсортированный массив. Для этого создается новый массив, в который будут последовательно добавляться элементы из левой и правой половин. При добавлении элементов выбирается наименьший элемент из двух половин и добавляется в новый массив. Этот шаг повторяется до тех пор, пока все элементы не будут добавлены в новый массив~\cite{merge-sort}.
\end{enumerate}

\section{Сложность алгоритма}
Обозначим размер сортируемого массива как $n$. 
Алгоритм сортировки слиянием разделяет массив на две половины до тех пор, пока размер каждой не станет равным 1. 
Это занимает $O(\log_{}(n))$ итераций.
На каждом шаге происходит слияние двух отсортированных половин, что занимает $O(n)$ времени. 
Так как слияние происходит на каждом шаге, общее время слияния будет $O(n \cdot \log_{}(n))$.
Таким образом, общее время выполнения алгоритма сортировки слиянием занимает $O(n \cdot \log_{}(n))$.

Алгоритм сортировки слиянием требует дополнительной памяти для хранения временных массивов при разделении и слиянии. 
Размер временных массивов равен размеру исходного массива~\cite{merge-sort}.

\section{Использование потоков}
В данной задаче возможно использование потоков на 2 этапе алгоритма. 
Создание потока возможно на каждый шаг рекурсии: в таком случае отдельный поток будет рекурсивно сортировать отдельную часть массива и затем выполнять ее слияние. Поток, запустивший следующий шаг рекурсии, выделяет на каждую половину массива отдельный поток и ждет окончания работы потоков, сортирующих 2 части массива. 

Так как данные массива изменяются только на шаге слияния и сортировки отдельных частей, а рекурсивная сортировка разбивает массив на непересекающиеся части, в использовании средств синхронизации в виде мьютекса нет необходимости.

\section*{Вывод}
В данной части работы был описан алгоритм сортировки слиянием и рассмотрено использование многопоточности в его реализации.




