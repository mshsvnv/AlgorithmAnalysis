\chapter*{\hfill{\centering  ВВЕДЕНИЕ}\hfill}
\addcontentsline{toc}{chapter}{ВВЕДЕНИЕ}

Многопоточность представляет собой способность процессора выполнять несколько задач или потоков одновременно, что обеспечивается операционной системой. Этот подход отличается от многопроцессорности, где каждый процессор выполняет отдельную задачу.

При последовательной реализации алгоритма только одно ядро процессора используется для выполнения программы. Однако при использовании параллельных вычислений~--- многопоточности~--- разные ядра могут одновременно решать независимые вычислительные задачи, что приводит к ускорению общего решения задачи~\cite{muti-thread}.

Целью данной лабораторной работы является получение навыков организации параллельного выполнения операций.

Для достижения поставленной цели необходимо выполнить следующие задачи:
\begin{enumerate}
	\item описать алгоритм сортировки слиянием;
	\item разработать версии приведенного алгоритма при использовании одного и нескольких потоков;
	\item определить средства программной реализации;
	\item реализовать разработанные алгоритмы;
	\item выполнить замеры процессорного времени работы различных реализаций алгоритма;
	\item провести сравнительный анализ зависимостей времени решения задачи от размерности входа и количества вспомогательных потоков.
\end{enumerate}