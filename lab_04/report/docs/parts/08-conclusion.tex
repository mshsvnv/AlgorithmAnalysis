\chapter*{\hfill{\centering  ЗАКЛЮЧЕНИЕ}\hfill}
\addcontentsline{toc}{chapter}{ЗАКЛЮЧЕНИЕ}

В результате исследования было получено, что при использовании 6 вспомогательных потоков при сортировке 50000 элементов требуется в 1.27 раз меньше времени для получения результата. 
При увеличении числа элементов в массиве
время получения результата с использованием нескольких потоков также увеличивается. При увлечении числа сортируемых элементов с 10000 до 70000, время получения результата при использовании одного потока  увеличилось в 7.4 раза.
При увлечении числа сортируемых элементов с 10000 до 70000, время получения результата при использовании 1 вспомогательного потока увеличилось в 10.8 раз. Наилучший результат при сортировке 50000 был получен при использовании 2 потоков, так как в таком случае минимальное время тратится на переключение контекста и потоки сортируют подмассивы максимального размера.


Поставленная цель была достигнута: получены навыки организации параллельного выполнения операций.

Для поставленной цели были выполнены все поставленные задачи.
\label{sec:targets}
\begin{enumerate}
	\item Описать алгоритм сортировки слиянием.
	\item Разработать версии  приведенного алгоритма, при использовании 1 потока и нескольких потоков.
	\item Определить средства программной реализации.
	\item Реализовать разработанные алгоритмы.
	\item Выполнить замеры процессорного времени работы различных реализаций алгоритма.
	\item Провести анализ времени получения отсортированных данных.
\end{enumerate}

 

