\chapter{Аналитическая часть}
В данной части работы будет описан алгоритм сортировки слиянием, 
а также рассмотрено использование многопоточности при его реализации.

\section{Описание алгоритма}
Сортировка слиянием~--- это эффективный алгоритм сортировки, который работает по принципу разделения и объединения~\cite{merge-sort}.

Этапы сортировки слиянием обычно включают в себя~\cite{merge-sort}:.
\begin{enumerate}
	\item Разделение: исходный массив разделяется на две (или более) равные части. 
	Этот процесс повторяется рекурсивно для каждой половины, пока длина подмассивов не станет равна 1;

	\item Слияние: отсортированные подмассивы объединяются в один, путем сравнения элементов и добавления их в результирующий массив в правильном порядке;

	\item Возврат: результат объединения возвращается как отсортированный массив;
\end{enumerate}
Ключевой момент заключается в рекурсивном разделении и слиянии массивов до тех пор, пока массивы не станут минимальной длины для сортировки.

\section{Сложность алгоритма}
Обозначим размер сортируемого массива как $n$. 
Алгоритм сортировки слиянием разделяет массив на две половины до тех пор, пока размер каждой  не станет равным 1. Это занимает $O(log(n))$ итераций.
На каждом шаге происходит слияние двух отсортированных половин, что занимает O(n) времени. Так как слияние происходит на каждом шаге, общее время слияния будет $O(n \cdot log(n))$.
Таким образом, общее время выполнения алгоритма сортировки слиянием составляет $O(n  \cdot log(n))$~\cite{merge-sort}.

Алгоритм сортировки слиянием требует дополнительной памяти для хранения временных массивов при разделении и слиянии. Размер временных массивов равен размеру исходного массива~\cite{merge-sort}.

\section{Использование потоков}
В данной задаче возможно использование потоков на 2 этапе алгоритма. Возможно создание  потока на каждый шаг рекурсии, в таком случае, 
отдельный поток будет рекурсивно сортировать отдельную часть массива  и затем выполнять ее слияние. 
Поток, запустивший следующий шаг рекурсии, выделяет на каждую половину массива отдельный поток и ждет окончания работы потоков, сортирующих 2 части массива. 
Так как данные массива изменяются только на шаге слияния и сортировки отдельных частей, а рекурсивная сортировка разбивает массив на непересекающиеся части, в использовании средств синхронизации в виде мьютекса \textbf{нет необходимости}.

\section*{Вывод}
В данной части работы был описан алгоритм сортировки слиянием и рассмотрено использование многопоточности в его реализации.






