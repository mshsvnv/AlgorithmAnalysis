\chapter{Конструкторская часть}
В данной части работы будут рассмотрены схемы алгоритмов различных реализаций сортировки слиянием.

\section{Требования к программному обеспечению}
К программе предъявлен ряд требований:
\begin{itemize}
	\item наличие интерфейса для выбора действий;
	\item динамическое выделение памяти под массив данных;
	\item работа с массивами и <<нативными>> потоками.
\end{itemize}

\section{Требования к временным замерам}
Процессорное время~--- это время, которое потратил процессор  на выполнение задачи.
Реальное время~--- время, прошедшее с начала выполнения задачи~\cite{time}. 

В данной работе, при использовании многопоточности возможно ожидание одними потоками выполнения других потоков. 
В это время поток не выполняет никаких действий, поэтому простой не влияет на процессорное время, однако это влияет на результирующее реальное время.
Таким образом для корректного сравнения различных реализаций по времени работы стоит замерять и сравнивать реальное время
выполнения реализаций алгоритмов.

\section{Схемы алгоритмов}
На рисунках~\ref{img:mergeSort}~--~\ref{img:merge} приведены схемы алгоритмов различных вариаций алгоритма сортировки слиянием

\includeimage
{mergeSort} 
{f} 
{H} 
{1\textwidth} 
{Схема алгоритма сортировки слиянием при использовании одного потока} 

\includeimage
{mergeSortMultiThread}
{f}
{H} 
{1\textwidth} 
{Схема алгоритма сортировки слиянием при использовании нескольких потоков}

\includeimage
{merge}
{f} 
{H} 
{0.9\textwidth} 
{Схема алгоритма слияния отсортированных последовательностей}

\newpage

Алгоритм слияния отсортированных последовательностей используется во всех рассматриваемых версиях алгоритма; в случае использования нескольких потоков, слияние будет происходить в отдельном потоке.

При использовании нескольких потоков (схема на рис.~\ref{img:mergeSortMultiThread}), число доступных потоков делится на 2 при каждом шаге рекурсии~--- в таком случае число доступных потоков поровну разделяется при каждом разбиении массива на подмассивы для слияния.

\section*{Вывод}
В данном разделе  были построены схемы рассматриваемых алгоритмов.
