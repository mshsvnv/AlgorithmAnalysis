\chapter{Аналитический раздел}
В данном разделе будет рассмотрена информация, касающаяся основ конвейерной обработки данных.
 
\section{Конвейерная обработка данных}
Конвейер~--- организация вычислений, при которой увеличивается количество выполняемых инструкций за единицу времени засчет использования принципов параллельности.

Конвейеризация в компьютерной обработке данных основана на разбиении выполнения функций на более мелкие этапы, называемые ступенями, и выделении отдельной аппаратуры для каждой из них.
Это позволяет организовать передачу данных от одного этапа к следующему, что увеличивает производительность за счет одновременного выполнения нескольких команд на различных ступенях конвейера.

Хотя конвейеризация не уменьшает время выполнения отдельной команды, она повышает пропускную способность процессора, что приводит к более быстрому выполнению программы по сравнению с простой, не конвейерной схемой~\cite{conway}.

\section{Разреженный строчный формат матрицы}

\section{Кольцевая КРМ~--~схема}
Схема Кнута~--~Рейнболдта~--~Местеньи или \textit{{Кольцевая КРМ~--~схема} является модификацией схемы Кнута.

\section{Описание алгоритмов}
В качестве операций, выполняющихся на конвейере, взяты следующие:
\begin{enumerate}
	\item 
\end{enumerate}

\section*{Вывод}
В данном разделе были рассмотрены алгоритм Кнута~--~Морриса~--~Пратта и его модификация.