\chapter{Исследовательский раздел}
В данном разделе будут приведены примеры работы программ, постанов-
ка исследования и сравнительный анализ алгоритмов на основе полученных данных.

\section{Технические характеристики}
Технические характеристики устройства, на котором выполнялись замеры по времени, представлены далее.
\begin{itemize}
	\item процессор: AMD Ryzen 5 5500U\,--\,2.10 ГГц;
	\item Оперативная память: 16 ГБайт;
	\item Операционная система: Windows 10 Pro 64-разрядная система версии 22H2.
\end{itemize}

При замерах времени ноутбук был включен в сеть электропитания и был нагружен только системными приложениями.

\section{Демонстрация работы программы}
На рисунке~\ref{img:prog_demo2} представлен пример результата работы программы. 
Пользователь, указывая соответствующие пункты меню, запускает последовательную обработку заявок, затем параллельное исполнение конвейера, затем выходит из программы. 

Лог программы при этом записывается в файл. 
На рисунке~\ref{img:prog_demo1} представлен пример лог–файла.
\includeimage
{prog_demo2}
{f}
{H}
{.8\textwidth}
{Демонстрация работы программы}

\includeimage
{prog_demo1}
{f}
{H}
{.6\textwidth}
{Пример файла с логом работы конвейера}

\section{Временные характеристики}
Для замеров времени использовалась функция получения значения системных часов \texttt{clock\_gettime()}~\cite{cpp-ctime}. 
Функция применялась два раза~--- в начале и в конце измерения времени, значения полученных временных меток вычитались друг из друга для получения времени выполнения программы.

Исследование временных характеристик реализаций алгоритмов производилось при изменении числа заявок от 10 до 100 с шагом 10 для матриц размером 10.

В таблице~\ref{tbl:time} представлены замеры времени выполнения двух реализаций конвейерной обработки в зависимости от количества заявок:
\begin{table}[ht]
	\small
	\begin{center}
		\begin{threeparttable}
			\caption{Результаты нагрузочного тестирования (в мкс)}
			\label{tbl:time}
			\begin{tabular}{|c|c|c|}
				\hline
				\multirow{2}{*}{\bfseries Кол-во заявок} & \multicolumn{2}{c|}{\bfseries Время, мкс} \\ \cline{2-3}
				& \bfseries Последовательный & \bfseries Параллельный
				\csvreader{csv/times.csv}{}
				{\\\hline \csvcoli & \csvcolii & \csvcoliii } \\
				\hline
			\end{tabular}
		\end{threeparttable}
	\end{center}
\end{table}

На рисунке~\ref{img:figure} приведен график результатов замеров для различных значений количества заявок.
\includeimage
{figure}
{f}
{H}
{1\textwidth}
{Результаты замеров времени работы реализации конвейерной обработки при варьировании числа заявок}

\section{Вывод}
В результате эксперимента было получено, что использование конвейерной обработки лучше по времени линейной реализации на 100 заявках примерно в 1.2 раза. 

В силу линейности графиков на рисунке~\ref{img:figure} можно сказать, что на достаточно большом количестве заявок выигрыш параллельной обработки над последовательной во времени в абсолютных единицах будет увеличиваться.


