\chapter{Технологический раздел}
В данном разделе будут описаны средства реализации программного обеспечения, а также представлены листинги и функциональные тесты.

\section{Средства реализации}
В качестве языка программирования, используемого при написании данной лабораторной работы, был выбран C++~\cite{cpp-lang}, так как в нем имеется контейнер \texttt{std::vector}, представляющий собой динамический массив данных произвольного типа, и библиотека \texttt{<ctime>}~\cite{cpp-ctime}, позволяющая производить замеры процессорного времени.
Также выбранный язяк программирования предоставляет возможность работы с потоками (класс \texttt{thread}~\cite{cpp-thread}) и мьютексами (класс \texttt{mutex}~\cite{cpp-mutex}).

\section{Сведения о модулях программы}
Данная программа разбита на следующие модули:
\begin{itemize}
	\item \texttt{main.cpp}~--- файл, содержащий точку входа в программу;
	\item \texttt{matrix.cpp}~--- файл содержит функции операций над матрицей и матрицей в РСФ;
	\item \texttt{read.cpp}~--- файл содержит функции чтения данных;
	\item \texttt{conveyor.cpp}~--- файл содержит функции конвейерной обработки;
	\item \texttt{meassure.cpp}~--- файл содержит функции, замеряющее процессорное время работы реализаций алгоритмов.
\end{itemize}
	
\section{Реализация алгоритмов}
На листингах~\ref{lst:gen_mtr.cpp}~--~\ref{lst:thread3.cpp} представлены реализации разрабатываемых алгоритмов.

\newpage 

\includelistingpretty
{gen_mtr.cpp}
{c++}
{Реализация алгоритма генерации данных для матрицы РСФ}

\newpage 

\includelistingpretty
{sum_mtr.cpp}
{c++}
{Реализация алгоритма сложения РСФ-матриц}

\newpage 

\includelistingpretty
{decom_mtr.cpp}
{c++}
{Реализация алгоритма распаковки матрицы}

\newpage 

\includelistingpretty
{linear.cpp}
{c++}
{Реализация последовательной конвейерной обработки}

\newpage 

\includelistingpretty
{parallel.cpp}
{c++}
{Реализация основного потока для конвейерной обработки, создающий вспомогательные потоки}

\newpage 

\includelistingpretty
{thread1.cpp}
{c++}
{Реализация вспомогательного потока, отвечающий за создание матриц РСФ}

\newpage 

\includelistingpretty
{thread2.cpp}
{c++}
{Реализация вспомогательного потока, отвечающий за сложение матриц РСФ}

\newpage 

\includelistingpretty
{thread3.cpp}
{c++}
{Реализация вспомогательного потока, отвечающий за распаковку матрицы}

\section*{Вывод}
В данном разделе была приведена информация о выбранных средствах для разработки алгоритмов. 
Были представлены листинги для каждой из реализаций работы конвейера и его трех стадий, а именно генерации данных для двух матриц РСФ, сложение двух матриц РСФ и распаковка матрицы РСФ в классическое матричное представление.