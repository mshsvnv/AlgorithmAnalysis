\chapter{Технологическая часть}
В данном разделе будут указаны средства реализации, листинг кода и функциональные тесты.

\section{Средства реализации}
Для реализации данной лабораторной работы был выбран язык \texttt{C++}~\cite{cpp-lang}, так как в нем есть стандартная библиотека \texttt{ctime}~\cite{cpp-lang}, которая позволяет производить замеры процессорного времени выполнения программы;

В качестве среды разработки был выбран \textit{Visual Studio Code}: он является кроссплатформенным и предоставляет полный набор интсрументов для проектирования и отладки кода.
 
\section{Сведения о модулях программы}
Данная программа разбита на следующие модули:

\begin{itemize}
	\item \texttt{main.cpp}~--- файл содержит точку входа в программу, из которой происходит вызов алгоритмов по разработанному интерфейсу;
	\item \texttt{array.cpp}~--- файл содержит реализацию класса \texttt{ArrayT};
	\item \texttt{sorts.cpp}~--- файл содержит функции алгоритомов сортировок;
	\item \texttt{measure.cpp}~--- файл содержит функции, замеряющие процессорное время выполнения алгоритмов сортировок;
\end{itemize}

\section{Реализация алгоритмов}
В листинге \ref{lst:radix} приведена реализация алгоритма поразрядной сортировки.

В листинге \ref{lst:shell} приведена реализация алгоритма сортировки Шелла.

В листинге \ref{lst:comb} приведена реализация алгоритма сортировки расческой.

\lstinputlisting[label=lst:radix, caption=Функция поразрядной сортировки, firstline=34,lastline=83]{../code/src/sorts.cpp}

\clearpage

\lstinputlisting[label=lst:shell, caption=Функция сортировки Шелла, firstline=110, lastline=128]{../code/src/sorts.cpp}

\lstinputlisting[label=lst:comb, caption=Функция сортировки расческой, firstline=88,lastline=104]{../code/src/sorts.cpp}

\clearpage

\section{Функциональные тесты}
В таблице \ref{tab:tests} приведены функциональные тесты для разработанных алгоритмов сортировки. Все тесты пройдены успешно.

\begin{table}[ht]
	\small
	\begin{center}
		\begin{threeparttable}
			\caption{Функциональные тесты}
			\label{tab:tests}
			\begin{tabular}{|c|c|c|c|c|c|}
				\hline
				\bfseries Массив
				& \bfseries Размер
				& \bfseries Ожидаемый рез.
				& \multicolumn{3}{c|}{\bfseries Фактический рез.} \\ \cline{4-6}
				& & & \bfseries Поразрядная & \bfseries Шелла & \bfseries Расческой \\
				\hline
				1 2 3 4  & 4 & 1 2 3 4 & 1 2 3 4 & 1 2 3 4 & 1 2 3 4 \\
				\hline
				4 3 2 1 & 4 & 4 3 2 1 & 4 3 2 1 & 4 3 2 1 & 4 3 2 1 \\
				\hline
				1 2 3 -9 & 4 & -9 1 2 3  & -9 1 2 3 & -9 1 2 3 & -9 1 2 3 \\
				\hline
				-5 -1 -3 -4 -5 & 5 & -5 -5 -4 -3 -1 & -5 -5 -4 -3 -1 & -5 -5 -4 -3 -1 \\
				\hline
			\end{tabular}	
		\end{threeparttable}	
	\end{center}
\end{table}

\section*{Вывод}
Были представлены листинги функций, функциональные тесты.
Также в данном разделе была представлена информация о выбранных средствах для разработки алгоритмов.