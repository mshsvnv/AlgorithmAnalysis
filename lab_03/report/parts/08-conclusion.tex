\chapter*{Заключение}
\addcontentsline{toc}{chapter}{Заключение}

Сортировка Шелла демонстрирует лучшую производительность при сортировке упорядоченных массивов. 
Поразрядная сортировка не эффективна из-за ограниченного числа уникальных разрядов, а сортировка расческой проводит сортировку слишком большими шагами, что увеличивает время выполнения из-за ненужных перемещений элементов. 

Поразрядная сортировка оказалась эффективной только при сортировке неотрицательных чисел с небольшим числом разрядов. 

Таким образом, сортировка Шелла продемонстрировала наилучшую производительность по сравнению с поразрядной сортировкой и сортировкой расческой, требуя при этом меньше памяти.

Цель данной лабораторной работы была достигнута, а именно были исследованы трудоемкости алгоритмов сортировки.

В результате выполнения лабораторной работы для достижения этой цели были выполнены следующие задачи:
\begin{enumerate}
    \item описаны следующие алгоритмы сортировки:
        \begin{itemize}
            \item поразрядная;
            \item расческой;
            \item Шелла;
        \end{itemize}
    \item релизованы описанные алгоритмы;
    \item дана оценка трудоемкости алгоритмов;
    \item дана оценка потребляемой памяти реализацими алгоритмов;
    \item проведены замеры времени выполнения алгоритмов;
\end{enumerate}

