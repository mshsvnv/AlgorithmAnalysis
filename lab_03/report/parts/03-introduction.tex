\chapter*{Введение}
\addcontentsline{toc}{chapter}{Введение}

Алгоритмы сортировки~--- это процессы упорядочивания элементов в определенном порядке. Эти алгоритмы могут быть использованы для сортировки различных типов данных, таких как числа, строки, структуры данных и другие.

Общие этапы алгоритмов сортировки включают:
\begin{enumerate}
    \item \textit{сравнение}: Ссравнение пар элементов для определения их относительного порядка;
    \item \textit{обмен или перемещение}: в зависимости от результата сравнения, элементы могут быть обменены местами или перемещены для достижения необходимого порядка.
    \item \textit{повторение}: этапы сравнения и обмена/перемещения повторяются, пока все элементы не будут упорядочены.
\end{enumerate}

В зависимости от конкретного алгоритма сортировки, могут существовать дополнительные этапы, такие как выбор опорного элемента (например, в быстрой сортировке), разделение массива на подмассивы (например, в сортировке слиянием), и другие действия, которые определяются спецификой конкретного алгоритма.\cite{knut}

\textbf{Целью} данной лабораторной работы является исследование трудоемкостей алгоритмов сортировки.

Для достижения поставленной цели необходимо выполнить следующие \textbf{задачи}:
\begin{enumerate}[label={\arabic*)}]
    \item описать следующие алгоритмы сортировки:
        \begin{itemize}
            \item поразрядная;
            \item расческой;
            \item Шелла;
        \end{itemize}
    \item релизовать описанные алгоритмы;
    \item дать оценку трудоемкости алгоритмов;
    \item дать оценку потребляемой памяти реализациями алгоритмов;
    \item провести замеры времени выполнения алгоритмов.
\end{enumerate}
