\chapter*{Введение}
\addcontentsline{toc}{chapter}{Введение}

Умножение матриц является основным инструментом линейной алгебры и имеет многочисленные применения в математике, физике, программировании~\cite{vinograd-haskell}. 

\textbf{Целью} данной лабораторной работы является исследование трудоемкостей алгоритмов сортировки.

Для достижения поставленной цели необходимо выполнить следующие \textbf{задачи}:
\begin{enumerate}[label={\arabic*)}]
    \item описать следующие алгоритмы сортировки:
        \begin{itemize}
            \item поразрядная;
            \item расческой;
            \item Шелла;
        \end{itemize}
    \item релизовать описанные алгоритмы;
    \item дать оценку трудоемкости алгоритмов;
    \item дать оценку потребляемой памяти реализациями алгоритмов;
    \item провести замеры времени выполнения алгоритмов.
\end{enumerate}
