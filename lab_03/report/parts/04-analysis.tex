\chapter{Аналитическая часть}
В данном разделе будут рассмотрены алгоритмы поразраядной сортировки, расческой и Шелла.

\section{Поразрядная сортировка}


\section{Сортировка расческой}


\section{Сортировка Шелла}
Метод предложен в 1959 году и назван по имени автора метода Дональда Шелла.
Состоит из прямого и обратного хода. 
Сравниваются и обмениваются не непосредственные соседи, а элементы, отстоящие на заданном расстоянии. 

Когда обнаружена перестановка, цепочка вторичных сравнений охватывает те элементы, которые входили в последовательность первичных просмотров.
Каждый последующий просмотр производится с уменьшенным шагом, на последнем просмотре шаг должен равняться 1. 
Можем использоватьследующую процедуру выбора шага. 
На первом просмотре шаг имеет значение $d = 2^{k} - 1$, где $k$ выбрано из условия $2^{k} \lt n \le 2^{k + 1}$. 
Новый просмотр производится с шагом $d = \frac{(d - 1)}{2}$. Сортировка заканчивается при $d = 0.\cite{shell}

\section*{Вывод}
В данном разделе были рассмотрены алгоритмы поразрядной сортировки, расческой и Шелла.