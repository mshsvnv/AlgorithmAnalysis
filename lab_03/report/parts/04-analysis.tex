\chapter{Аналитическая часть}
В данном разделе будут рассмотрены алгоритмы поразраядной сортировки, расческой и Шелла.

\section{Поразрядная сортировка}


\section{Сортировка расческой}
Основная идея «расчёски» в том, чтобы первоначально брать достаточно большое расстояние между сравниваемыми элементами и по мере упорядочивания массива сужать это расстояние вплоть до минимального. 
Таким образом, мы как бы причёсываем массив, постепенно разглаживая на всё более аккуратные пряди. 

Первоначальный разрыв между сравниваемыми элементами лучше брать с учётом специальной величины, называемой фактором уменьшения, оптимальное значение которой равно примерно 1,247. 
Сначала расстояние между элементами максимально, то есть равно размеру массива минус один. Затем, пройдя массив с этим шагом, необходимо поделить шаг на фактор уменьшения и пройти по списку вновь. Так продолжается до тех пор, пока разность индексов не достигнет единицы. В этом случае сравниваются соседние элементы как и в сортировке пузырьком, но такая итерация одна.

Оптимальное значение фактора уменьшения 
1,247
33...
=
1
1
−
�
−
Φ{\displaystyle 1{,}24733...={\tfrac {1}{1-e^{-\Phi }}}}, где 
�
e — основание натурального логарифма, а 
Φ
=
1,618
03...
{\displaystyle \Phi =1{,}61803...} — золотое сечение.

\section{Сортировка Шелла}
Метод предложен в 1959 году и назван по имени автора метода Дональда Шелла.
Состоит из прямого и обратного хода. 
Сравниваются и обмениваются не непосредственные соседи, а элементы, отстоящие на заданном расстоянии. 

Когда обнаружена перестановка, цепочка вторичных сравнений охватывает те элементы, которые входили в последовательность первичных просмотров.
Каждый последующий просмотр производится с уменьшенным шагом, на последнем просмотре шаг должен равняться 1. 
Можем использоватьследующую процедуру выбора шага. 
На первом просмотре шаг имеет значение $d = 2^{k} - 1$, где $k$ выбрано из условия $2^{k} \lt n \le 2^{k + 1}$. 
Новый просмотр производится с шагом $d = \frac{(d - 1)}{2}$. Сортировка заканчивается при $d = 0.\cite{shell}

\section*{Вывод}
В данном разделе были рассмотрены алгоритмы поразрядной сортировки, расческой и Шелла.