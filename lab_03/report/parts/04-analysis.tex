\chapter{Аналитическая часть}
В данном разделе будут рассмотрены алгоритмы поразраядной сортировки, расческой и Шелла.

\section{Поразрядная сортировка}
Поразрядная сортировка~--- это алгоритм сортировки, который сортирует целые числа по их разрядам, начиная с наименьшего разряда и заканчивая наибольшим. 
Каждое число разбивается на отдельные цифры, затем сравниваются и перемещаются в нужные позиции. 
Этот процесс повторяется для всех разрядов, пока числа не будут упорядочены.

Поразрядная сортировка с сортировкой подсчета состоит из следующих этапов:
\begin{enumerate}
	\item нахождение максимального числа в массиве, чтобы определить количество разрядов;
	\item инициализация массива подсчета и заполнение его нулями для каждого возможного значения разряда;
	\item подсчет количества элементов с каждым разрядом путем прохода по массиву и увеличения соответствующих счетчиков;
	\item преобразование массива подсчета, чтобы он содержал фактические позиции для конкретного разряда, путем сложения предыдущих счетчиков;
	\item создание выходного массива той же длины, что и входной;
	\item размещение элементов в правильные позиции в выходном массиве согласно их разрядам, уменьшение счетчиков в массиве подсчета после каждого размещения;
	\item повторение шагов 3-6 для каждого разряда, начиная с младшего разряда и двигаясь к старшему;
\end{enumerate}

Инициализация массива подсчета и заполнение его нулями для каждого возможного значения разряда.
Подсчет количества элементов с каждым разрядом путем прохода по массиву и увеличения соответствующих счетчиков.
Преобразование массива подсчета, чтобы он содержал фактические позиции для конкретного разряда, путем сложения предыдущих счетчиков.
Создание выходного массива той же длины, что и входной.
Размещение элементов в правильные позиции в выходном массиве согласно их разрядам, уменьшение счетчиков в массиве подсчета после каждого размещения.
Повторение шагов 3-6 для каждого разряда, начиная с младшего разряда и двигаясь к старшему.

\section{Сортировка расческой}
Основная идея «расчески» в том, чтобы первоначально брать достаточно большое расстояние между сравниваемыми элементами и по мере упорядочивания массива сужать это расстояние вплоть до минимального. 
Таким образом, мы как бы причесываем массив, постепенно разглаживая на все более аккуратные пряди. 

Первоначальный разрыв между сравниваемыми элементами лучше брать с учетом специальной величины, называемой фактором уменьшения, оптимальное значение которой равно примерно 1.247. 
Сначала расстояние между элементами максимально, то есть равно размеру массива минус один. 
Затем, пройдя массив с этим шагом, необходимо поделить шаг на фактор уменьшения и пройти по списку вновь. 
Так продолжается до тех пор, пока разность индексов не достигнет единицы. 
В этом случае сравниваются соседние элементы как и в сортировке пузырьком, но такая итерация одна.

Оптимальное значение фактора уменьшения $1.24733\dots = \frac{1}{1 - e^{-\Phi}}$, где $e$~--- основание натурального логарифма, а ${\Phi = 1.61803...}$~--- золотое сечение~\cite{article_sorts}.

\section{Сортировка Шелла}
Сортировка Шелла состоит из прямого и обратного хода. 
Сравниваются и обмениваются не непосредственные соседи, а элементы, отстоящие на заданном расстоянии. 

Когда обнаружена перестановка, цепочка вторичных сравнений охватывает те элементы, которые входили в последовательность первичных просмотров.
Каждый последующий просмотр производится с уменьшенным шагом, на последнем просмотре шаг должен равняться 1. 
Можем использовать следующую процедуру выбора шага. 
На первом просмотре шаг имеет значение $d = 2^{k} - 1$, где $k$ выбрано из условия: \[2^{k} < n \le 2^{k + 1}.\] 
Новый просмотр производится с шагом $d = \frac{(d - 1)}{2}$. 
Сортировка заканчивается при $d = 0$~\cite{shell}.

\section*{Вывод}
В данном разделе были рассмотрены алгоритмы поразрядной сортировки, расческой и Шелла.