\chapter*{Введение}
\addcontentsline{toc}{chapter}{Введение}

В данной лабораторной работе будут рассмотрены расстояния Левенштейна и Дамерау-Левенштейна. 

\textbf{Расстояние Левенштейна} — это метрика, используемая для измерения разницы между двумя строками. Он вычисляет минимальное количество односимвольных изменений (вставок, удалений или замен), необходимых для преобразования одной строки в другую.

Расстояние Левенштейна обычно используется в различных областях, в том числе:
\begin{enumerate}
    \item \textbf{Проверка орфографии:} выявление и исправление ошибок для слов на основе их расстояния Левенштейна.
    \item \textbf{Анализ последовательности ДНК:} измерение сходства между последовательностями ДНК позволяет исследователям сравнивать и анализировать генетические данные.
    \item \textbf{Обработка естественного языка (НЛП):} используется в таких задачах, как классификация текста, поиск информации и машинный перевод, для определения сходства между текстами.
\end{enumerate}

\textbf{Расстояние Дамерау-Левенштейна} является расширением расстояния Левенштейна, которое не учитывает транспозиции. Дополнительная операция транспонирования на расстоянии Дамерау-Левенштейна позволяет обрабатывать случаи, когда символы меняются местами или переупорядочиваются.

\textbf{Целью} данной лабораторной работы является изучение, реализация и исследование алгоритмов поиска расстояний Левенштейна и Дамерау-Левенштейна.

Необходимо выполнить следующие \textbf{задачи}:
\begin{enumerate}[label={\arabic*)}]
        \item изучить алгоритмы Левенштейна и Дамерау-Левенштейна для нахождения расстояния между строкж;
        \item применить методы динамического программирования при реализации алгоритмов; 
	\item создать программное обеспечение, реализующее следующие алгоритмы:
	\begin{itemize}[label=---]
		\item нерекурсивный алгоритм поиска расстояния Левенштейна;
		\item нерекурсивный алгоритм поиска расстояния Дамерау-Левенштейна;
		\item рекурсивный алгоритм поиска расстояния Дамерау-Левенштейна;
		\item рекурсивный алгоритм поиска расстояния Дамерау-Левенштейна с кешированием.
	\end{itemize}
	\item провести сравнительный анализ алгоритмов определения расстояния между строками по затрачиваемым ресурсам (времении, памяти);
        \item описать и обосновать полученные результаты в отчете.
\end{enumerate}