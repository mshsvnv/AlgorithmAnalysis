\chapter{Аналитическая часть}
\section{Расстояние Левенштейна}

\textbf{Расстояние Левенштейна}~--- это минимальное количество редакторских операций вставки, замены и удаления, которые необходимо выполнить для преобразования одной строки в другую~\cite{levenshtein}. 

Каждая операция имеет свою цену($w$). Редакционное предписание~--- это последовательность операций с сумарной минимальной стоимостью, которую необходимо выполнить для получения из первой строки вторую. Эта цена и есть искомое расстояние Левенштейна.

Введем следующие обозначения:
\begin{enumerate}[label=\arabic*)]
    \item \textbf{I}~(от англ. insert)~--- вставка ($w(\lambda, b) = 1$);
    \item \textbf{R}~(от англ. replace)~--- замена ($w(a, b) = 1$, $a \neq b$);
    \item \textbf{D}~(от англ. delete)~--- удаление ($w(a, \lambda) = 1$).
\end{enumerate}

Также рассмотрим функцию $D(i, j)$: ее значением является
редакционное расстояние между строками $S_1[1...i]$ и $S_2[1...j]$.

Расстояние Левенштейна между двумя строками $S_{1}$ и $S_{2}$ (длиной $M$ и $N$ соответственно) рассчитывается по следующей рекуррентной формуле:

\begin{equation}
	\label{eq:L}
	D(i, j) =
	\begin{cases}
		0, &\text{i = 0, j = 0}\\
		i, &\text{j = 0, i > 0}\\
		j, &\text{i = 0, j > 0}\\
		min \begin{cases}
			D(i, j - 1) + 1,\\
			D(i - 1, j) + 1,\\
			D(i - 1, j - 1) +  m(S_{1}[i], S_{2}[j]), \\
		\end{cases}
		&\text{i > 0, j > 0}
	\end{cases}
\end{equation}
где сравнение символов строк $S_1$ и $S_2$ рассчитывается таким образом:
\begin{equation}
	\label{eq:m}
	m(a, b) = \begin{cases}
		0 &\text{если a = b,}\\
		1 &\text{иначе.}
	\end{cases}
\end{equation}

\section{Расстояние Дамерау~---~Левенштейна}

\textbf{Расстояние Дамерау~---~Левенштейна}~--- это мера разницы двух строк, определяемая как минимальное количество операций вставки, удаления, замены и транспозиции (перестановки двух соседних символов), необходимых для перевода одной строки в другую~\cite{analysis-lev-damlev}. Является расширением расстояния Левенштейна, поскольку помимо трех базовых операций содержит еще операцию транспозиции $T$ (от англ. transposition).

Расстояние Дамерау~---~Левенштейна определятся следующей рекуррентной формуле:

\begin{multline}
    D(m, n) =\\ =
    \begin{cases}
        0, &\text{i = 0, j = 0}\\
        i, &\text{j = 0, i > 0}\\
        j, &\text{i = 0, j > 0}\\
        \min
        \begin{cases}
            D(i, j - 1) + 1,\\
            D(i - 1, j) + 1,\\
            D(i - 1, j - 1),\\
            D(i - 2, j - 2) + 1,
        \end{cases} 
        &\begin{aligned}
            & \text{если $i, j > 1$}, \\
            & S_{1}[i] = S_{2}[j - 1], \\
            & S_{1}[i - 1] = S_{2}[j],
        \end{aligned} \\
        \min
        \begin{cases}
            D(i - 1, j) + 1, \\
            D(i, j - 1) + 1, \\
            D(i - 1, j - 1) + \text{m}(S_1[i], S_2[j])
        \end{cases} &\text{иначе.}
    \end{cases}
    \label{eqn:recur-damlev}
\end{multline}

\section*{Вывод}
В данном разделе были рассмотрены алгоритмы нахождения расстояния Левенштейна и Дамерау~---~Левенштейна.
Формулы для вычисления этих расстояний задаются \textbf{рекуррентно}, поэтому алгоритмы для нахождения их расстояний можно реализовать как \textit{итеративно}, так и \textit{рекурсивно}.
