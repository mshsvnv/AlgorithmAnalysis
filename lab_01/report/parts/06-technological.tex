\chapter{Технологическая часть}

В данном разделе будут приведены средства реализации, листинг кода и функциональные тесты.

\section{Средства реализации}

Для реализации данной лабораторной работы был выбран язык $C++$ \cite{cpp-lang}, так как в нем есть стандартная библиотека $chrono$, которая позволяет произвести замеры процессорного времени выполнения программы, а так же наличие типа данных $std::wstring$, позволяющего хранить как кириллические символы, так и латинские;
В качестве среды разработки был выбран $Visual Studio Code$: он является кроссплатформенным и предоставляет полный функционал для проектирования и отладки кода.
 
\section{Сведения о модулях программы}

Данная программа разбита на следующие модули.

\begin{itemize}
	\item \texttt{main.cpp} --- файл, содержащий точку входа в программу, из которой происходит вызов алгоритмов по разработанному интерфейсу.
	\item \texttt{algorithms.cpp} --- файл содержит функции поиска расстояния Левенштейна и Дамерау-Левенштейна.
	\item \texttt{matrix.cpp} --- файл содержит функции динамического выделения и очищения памяти для матрицы, а так же ее вывод на экран.
	\item \texttt{measure.cpp} --- файл содержит функции, замеряющее процессорное время для алгоритмов поиска расстояния Левенштейна и Дамерау-Левенштейна.
\end{itemize}

\section{Реализация алгоритмов}

В листингах \ref{lst:lev_mtr} -- \ref{lst:dameray_lev_rec_hash} приведены реализации алгоритмов поиска расстояний Левенштейна (только нерекурсивный алгоритм) и \break Дамерау-Левенштейна (нерекурсивный, рекурсивный и рекурсивный с кешированием).

\begin{lstlisting}[label=lst:lev_mtr,caption=Функция нахождения расстояния Левенштейна с использованием матрицы (начало)]
int notRecursiveLev(wstring &word1, wstring &word2, bool print) {
    
    int len1 = word1.length();
    int len2 = word2.length();

    int** mtr = allocateMtr(len2 + 1, len1 + 1);

    if (!mtr)
        return 0;

    for (int i = 0; i <= len2; ++i) {

        for (int j = 0; j <= len1; ++j) {

            if (i == 0)
                mtr[i][j] = j;
            else if (j == 0)
                mtr[i][j] = i;
            else {
                int dif = (word1[j - 1] == word2[i - 1]) ? 0 : 1;

                mtr[i][j] = min(mtr[i - 1][j] + 1, 
                               min(mtr[i][j - 1] + 1, mtr[i - 1][j - 1] + dif));
            }
        }
    }
\end{lstlisting}

\begin{lstlisting}[label=lst:lev_mtr,caption=Функция нахождения расстояния Левенштейна с использованием матрицы (окончание)]
    if (print)  
        printMtr(mtr, word1, word2);

    int res = mtr[len2][len1];
    freeMtr(mtr, len2 + 1);

    return res;
}
\end{lstlisting}

\begin{lstlisting}[label=lst:dameray_lev_rec,caption=Функция нахождения расстояния Дамерау-Левенштейна с использованием матрицы (начало)]
int notRecursiveDamLev(wstring &word1, wstring &word2, bool print) {

    int len1 = word1.length();
    int len2 = word2.length();

    int** mtr = allocateMtr(len2 + 1, len1 + 1);

    if (!mtr)
        return 0;

    for (int i = 0; i <= len2; ++i) {

        for (int j = 0; j <= len1; ++j) {

            if (i == 0)
                mtr[i][j] = j;
            else if (j == 0)
                mtr[i][j] = i;
            else {
                int dif = (word1[j - 1] == word2[i - 1]) ? 0 : 1;

                mtr[i][j] = min(mtr[i - 1][j] + 1, 
                                min(mtr[i][j - 1] + 1, mtr[i - 1][j - 1] + dif));

\end{lstlisting}

\begin{lstlisting}[label=lst:dameray_lev_rec,caption=Функция нахождения расстояния Дамерау-Левенштейна с использованием матрицы (окончание)]
                if (word1[j - 2] == word2[i - 1] && word1[j - 1] == word2[i - 2]) 
                    mtr[i][j] = min(mtr[i][j], mtr[i - 2][j - 2] + 1);
            }
        }
    }

    if (print)  
        printMtr(mtr, word1, word2);

    int res = mtr[len2][len1];
    freeMtr(mtr, len2 + 1);

    return res;
}
\end{lstlisting}

\clearpage

\begin{lstlisting}[label=lst:dameray_lev_mtr,caption=Функция нахождения расстояния Дамерау-Левенштейна рекурсивно]
int recursive(wstring &word1, wstring &word2, int ind1, int ind2) {

    if (min(ind1, ind2) == 0)
        return max(ind1, ind2);

    int dif = (word1[ind1 - 1] == word2[ind2 - 1]) ? 0 : 1;

    int res = min(recursive(word1, word2, ind1 - 1, ind2 - 1) + dif,
                  min(recursive(word1, word2, ind1 - 1, ind2) + 1, 
                      recursive(word1, word2, ind1, ind2 - 1) + 1));
                      
    if (word1[ind1 - 1] == word2[ind2 - 2] && word1[ind1 - 2] == word2[ind2 - 1])
        res = min(res, recursive(word1, word2, ind1 - 2, ind2 - 2) + 1);

    return res;
}
\end{lstlisting}

\clearpage

\begin{lstlisting}[label=lst:dameray_lev_rec_hash,caption=Функция нахождения расстояния Дамерау-Левенштейна рекурсивно c кешированием]
int recursiveCash(wstring &word1, wstring &word2, int ind1, int ind2, mapT& dict) {

    int key = ind1 + ind2 * (1 + word2.length());

    if (dict[key])
        return dict[key];

    if (min(ind1, ind2) == 0)
        return dict[key] = max(ind1, ind2);

    int dif = (word1[ind1 - 1] == word2[ind2 - 1]) ? 0 : 1;

    int res = min(recursiveCash(word1, word2, ind1 - 1, ind2 - 1, dict) + dif,
                  min(recursiveCash(word1, word2, ind1 - 1, ind2, dict) + 1, 
                      recursiveCash(word1, word2, ind1, ind2 - 1, dict) + 1));

    if (ind1 > 1 && ind2 > 1 && word1[ind1 - 1] == word2[ind2 - 2] && word1[ind1 - 2] == word2[ind2 - 1])
        res = min(res, recursiveCash(word1, word2, ind1 - 2, ind2 - 2, dict) + 1);

    dict[key] = res;

    return dict[key];
}
\end{lstlisting}

\clearpage

\section{Функциональные тесты}

В таблице \ref{tbl:func_tests} приведены функциональные тесты для алгоритмов вычисления расстояний Левенштейна и Дамерау—Левенштейна. Все тесты пройдены успешно.

\begin{table}[ht]
	\small
	\begin{center}
		\begin{threeparttable}
		\caption{Функциональные тесты}
		\label{tbl:func_tests}
		\begin{tabular}{|c|c|c|c|c|c|}
			\hline
			\multicolumn{2}{|c|}{\bfseries Входные данные}
			& \multicolumn{4}{c|}{\bfseries Расстояние и алгоритм} \\ 
			\hline 
			&
			& \multicolumn{1}{c|}{\bfseries Левенштейна} 
			& \multicolumn{3}{c|}{\bfseries Дамерау-Левенштейна} \\ \cline{3-6}
			
			\bfseries Строка 1 & \bfseries Строка 2 & \bfseries Итеративный & \bfseries Итеративный
			
			& \multicolumn{2}{c|}{\bfseries Рекурсивный} \\ \cline{5-6}
			& & & & \bfseries Без кеша & \bfseries С кешом \\
			\hline
			a & b & 1 & 1 & 1 & 1 \\
			\hline
			a & a & 0 & 0 & 0 & 0 \\
			\hline
			кот & скат & 2 & 2 & 2 & 2 \\
			\hline
			кот & кто & 2 & 1 & 1 & 1 \\
			\hline
			Австралия & Австрия & 2 & 2 & 2 & 2 \\
			\hline
			кот & ток & 2 & 2 & 2 & 2 \\
			\hline
			слон & слоны & 1 & 1 & 1 & 1 \\
			\hline
		\end{tabular}	
		\end{threeparttable}
	\end{center}
\end{table}

\section*{Вывод}
Были реализованы и протестированы алгоритмы поиска расстояния Левенштейна итеративно, а также поиска расстояния Дамерау–Левенштейна итеративно, рекурсивно и рекурсивного с кеширования. Проведено тестирование реализаций алгоритмов.
