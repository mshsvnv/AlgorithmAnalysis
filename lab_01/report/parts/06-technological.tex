\chapter{Технологическая часть}

В данном разделе будут приведены требования к программному обеспечению, средства реализации, листинг кода и функциональные тесты.

\section{Средства реализации}

Для реализации данной лабораторной работы был выбран язык $C++$ \cite{cpp-lang}. Данный выбор обусловлен наличием у языка встроенного модуля $ctime$ измерения процессорного времени и типа данных, позволяющего хранить как кириллические символы, так и латинские --- $std::wstring$, что позволит удовлетворить третьему требованию из п.\ref{section:requirements} соответствуют выдвинутым техническим требованиям.

Время работы было замерено с помощью функции \textit{clock()} из библиотеки \textit{сtime}~\cite{cpp-lang-time}
 
\clearpage
 
\section{Сведения о модулях программы}

Данная программа разбита на следующие модули.

\begin{itemize}
	\item \texttt{main.cpp} --- файл, содержащий точку входа в программу, из которой происходит вызов алгоритмов по разработанному интерфейсу.
	\item \texttt{algorithms.cpp} --- файл содержит функции поиска расстояния Левенштейна и Дамерау-Левенштейна.
	\item \texttt{allocate.cpp} --- файл содержит функции динамического выделения и очищения памяти для матрицы.
	\item \texttt{print\_mtr\_lev.cpp} --- файл содержит функцию вывода матрицы для итеративных алгоритмов поиска расстояния Левенштейна и Дамерау-Левенштейна, включая строки.
	\item \texttt{cpu\_time.cpp} --- файл содержит функции, замеряющее процессорное время алгоритмов поиска расстояния Левенштейна и \newline Дамерау-Левенштейна.
	\item \texttt{memory.cpp} --- файл содержит функции, замеряющее память итеративного и рекурсивного алгоритмов поиска расстояния Левенштейна.
\end{itemize}

\section{Реализация алгоритмов}

В листингах \ref{lst:lev_mtr} -- \ref{lst:dameray_lev_rec_hash} приведены реализации алгоритмов поиска расстояний Левенштейна (только нерекурсивный алгоритм) и Дамерау-Левенштейна (нерекурсивный, рекурсивный и рекурсивный с кешированием).
В листингах \ref{lst:allocate_mtr} -- \ref{lst:print_mtr} приведены реализации алгоритмов выделения памяти под матрицу и вывод матрицы.

\clearpage

\begin{lstlisting}[label=lst:lev_mtr,caption=Функция нахождения расстояния Левенштейна с использованием матрицы]
int lev_mtr(wstring &str1, wstring &str2, bool print ) {
	size_t n = str1.length();
	size_t m = str2.length();
	int **mtr = malloc_mtr(n + 1, m + 1);
	int res = 0;
	for (int i = 0; i <= n; i++)
		for (int j = 0; j <= m; j++)
			if (i == 0 && j == 0)
				mtr[i][j] = 0;
			else if (i > 0 && j == 0)
				mtr[i][j] = i;
			else if (j > 0 && i == 0)
				mtr[i][j] = j;
	else {
		int change = 0;
		if (str1[i - 1] != str2[j - 1])
			change = 1;
		
		mtr[i][j] = std::min(mtr[i][j - 1] + 1,
					std::min(mtr[i - 1][j] + 1,
							 mtr[i - 1][j - 1] + change));
	}
	if (print)
		print_mtr_lev(str1, str2, mtr, n, m);
	res = mtr[n][m];
	free_mtr(mtr, n);
	return res;
}
\end{lstlisting}

\clearpage

\begin{lstlisting}[label=lst:dameray_lev_rec,caption=Функция нахождения расстояния Дамерау-Левенштейна с использованием матрицы]
int dameray_lev_mtr(wstring &str1, wstring &str2, bool print)
{
	size_t n = str1.length();
	size_t m = str2.length();
	int **mtr = malloc_mtr(n + 1, m + 1);
	int res = 0;
	
	for (int i = 0; i <= n; i++)
		for (int j = 0; j <= m; j++) {
			if (i == 0 && j == 0)
				mtr[i][j] = 0;
			else if (i > 0 && j == 0)
				mtr[i][j] = i;
			else if (j > 0 && i == 0)
				mtr[i][j] = j;
			else {
				int change = 0;
				if (str1[i - 1] != str2[j - 1])
					change = 1;
				
				mtr[i][j] = min(mtr[i][j - 1] + 1,
							min(mtr[i - 1][j] + 1,
								mtr[i - 1][j - 1] + change));
				
				if (i > 1 && j > 1 &&
				str1[i - 1] == str2[j - 2] &&
				str1[i - 2] == str2[j - 1])
					mtr[i][j] = min(mtr[i][j], mtr[i - 2][j - 2] + 1);
			}
	}
	
	if (print)
		print_mtr_lev(str1, str2, mtr, n, m);
	res = mtr[n][m];
	free_mtr(mtr, n);
	
	return res;
}
\end{lstlisting}

\clearpage

\begin{lstlisting}[label=lst:dameray_lev_mtr,caption=Функция нахождения расстояния Дамерау-Левенштейна рекурсивно]
int dameray_lev_rec_t(wstring &str1, wstring &str2, size_t n, size_t m) {
	if (n == 0)
		return m;
	if (m == 0)
		return n;
	
	int change = 0;
	int res = 0;
	if (str1[n - 1] != str2[m - 1])
		change = 1;
	
	res = min(dameray_lev_rec_t(str1, str2, n, m - 1) + 1,
		  min(dameray_lev_rec_t(str1, str2, n - 1, m) + 1,
		      dameray_lev_rec_t(str1, str2, n - 1, m - 1) + change));
	
	if (n > 1 && m > 1 &&
	str1[n - 1] == str2[m - 2] &&
	str1[n - 2] == str2[m - 1])
		res = std::min(res, dameray_lev_rec_t(str1, str2, n - 2, m - 2) + 1);
	return res;
}

int dameray_lev_rec(wstring &str1, wstring &str2)
{
	return dameray_lev_rec_t(str1, str2, str1.length(), str2.length());
}
\end{lstlisting}

\clearpage

\begin{lstlisting}[label=lst:dameray_lev_rec_hash,caption=Функция нахождения расстояния Дамерау-Левенштейна рекурсивно c кешированием]
int dameray_lev_rec_hash_t(wstring &str1, wstring &str2, int **mtr, size_t n, size_t m) {
	if (n == 0)
		return mtr[n][m] = m;
	if (m == 0)
		return mtr[n][m] = n;
	int change = 0;
	if (str1[n - 1] != str2[m - 1])
		change = 1;
	mtr[n][m] = min(dameray_lev_rec_hash_t(str1, str2, mtr, n, m - 1) + 1,
				min(dameray_lev_rec_hash_t(str1, str2, mtr, n - 1, m) + 1,
					dameray_lev_rec_hash_t(str1, str2, mtr, n - 1, m - 1) + change));
	if (n > 1 && m > 1 &&
	str1[n - 1] == str2[m - 2] &&
	str1[n - 2] == str2[m - 1])
		mtr[n][m] = min(mtr[n][m], dameray_lev_rec_hash_t(str1, str2, mtr, n - 2, m - 2) + 1);
	return mtr[n][m];
}

int dameray_lev_rec_hash(wstring &str1, wstring &str2, bool print) {
	size_t n = str1.length();
	size_t m = str2.length();
	int **mtr = malloc_mtr(n + 1, m + 1);
	for (int i = 0; i <= n; i++)
		for (int j = 0; j <= m; j++) {
			mtr[i][j] = -1;
		}
	int res = dameray_lev_rec_hash_t(str1, str2, mtr, n, m);
	if (print)
		print_mtr_lev(str1, str2, mtr, n, m);
	free_mtr(mtr, n);
	return res;
}
\end{lstlisting}

\clearpage

\begin{lstlisting}[label=lst:allocate_mtr,caption=Функции динамического выделения и очищения памяти под матрицу]
void free_mtr(int **mtr, std::size_t n) {
	if (mtr != nullptr)
	{
		for (std::size_t i = 0; i < n; i++)
		if (mtr[i] != nullptr)
			free(mtr[i]);
		free(mtr);
	}
}

int **malloc_mtr(std::size_t n, std::size_t m)
{
	if (n == 0)
	return nullptr;
	
	int **mtr = static_cast<int **>(malloc(n * sizeof(int *)));
	if (mtr != nullptr)
	for (std::size_t i = 0; mtr[i] != nullptr && i < n; i++) {
		mtr[i] = static_cast<int *>(malloc(m * sizeof(int)));
		if (mtr[i] == nullptr)
			free_mtr(mtr, n);
	}
	
	return mtr;
}
\end{lstlisting}

\clearpage

\begin{lstlisting}[label=lst:print_mtr,caption=Функции вывода матрицы для алгоритмов поиска расстояния Левенштейна и Дамерау-Левенштейна]
void print_mtr_lev(std::wstring str1, std::wstring str2,
int **mtr, std::size_t n, std::size_t m)
{
	for(std::size_t i = 0; i <= n + 1; i++)
	{
		for(std::size_t j = 0; j <= m + 1; j++)
		{
			if (i == 0 && j == 0)
				std::wcout << "  ";
			else if (i == 0)
			if (j == 1)
				std::wcout << "- ";
			else
				std::wcout << str2[j - 2] << " ";
			else if (j == 0)
			if (i == 1)
				std::wcout << "- ";
			else
				std::wcout << str1[i - 2] << " ";
			else
				std::wcout << mtr[i - 1][j - 1] << " ";
		}
		std::wcout << std::endl;
	}
}
\end{lstlisting}

\clearpage

\section{Функциональные тесты}

В таблице \ref{tbl:func_tests} приведены функциональные тесты для алгоритмов вычисления расстояний Левенштейна и Дамерау—Левенштейна. Все тесты пройдены успешно.

\begin{table}[ht]
	\small
	\begin{center}
		\begin{threeparttable}
		\caption{Функциональные тесты}
		\label{tbl:func_tests}
		\begin{tabular}{|c|c|c|c|c|c|}
			\hline
			\multicolumn{2}{|c|}{\bfseries Входные данные}
			& \multicolumn{4}{c|}{\bfseries Расстояние и алгоритм} \\ 
			\hline 
			&
			& \multicolumn{1}{c|}{\bfseries Левенштейна} 
			& \multicolumn{3}{c|}{\bfseries Дамерау-Левенштейна} \\ \cline{3-6}
			
			\bfseries Строка 1 & \bfseries Строка 2 & \bfseries Итеративный & \bfseries Итеративный
			
			& \multicolumn{2}{c|}{\bfseries Рекурсивный} \\ \cline{5-6}
			& & & & \bfseries Без кеша & \bfseries С кешом \\
			\hline
			a & b & 1 & 1 & 1 & 1 \\
			\hline
			a & a & 0 & 0 & 0 & 0 \\
			\hline
			кот & скат & 2 & 2 & 2 & 2 \\
			\hline
			друзья & рдузия & 3 & 2 & 2 & 2 \\
			\hline
			вагон & гонки & 4 & 4 & 4 & 4 \\
			\hline
			бар & раб & 2 & 2 & 2 & 2 \\
			\hline
			слон & слоны & 1 & 1 & 1 & 1 \\
			\hline
		\end{tabular}	
		\end{threeparttable}
	\end{center}
\end{table}

\section*{Вывод}

Были реализованы алгоритмы поиска расстояния Левенштейна итеративно, а также поиска расстояния Дамерау–Левенштейна итеративно, рекурсивно и рекурсивного с кеширования. Проведено тестирование реализаций алгоритмов.
