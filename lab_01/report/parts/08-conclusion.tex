\chapter*{Заключение}
\addcontentsline{toc}{chapter}{Заключение}

Цель данной лабораторной работы была достигнута, а именно были изучены, реализованы и исследованы алгоритмы поиска расстояний Левенштейна и Дамерау~---~Левенштейна.

В результате выполнения лабораторной работы для достижения этой цели были выполнены следующие задачи:
\begin{enumerate}
    \item описаны алгоритмы поиска расстояния Левенштейна и Дамерау~---~Левенштейна;
    \item разработаны и реализованы соответствующие алгоритмы;
    \item создано программное обеспечение, позволяющее протестировать реализованные алгоритмы;
    \item проведен сравнительный анализ процессорного времени выполнения реализованных алгоритмов:
    \begin{itemize}
        \item при малых длинах строк $(< 5)$ рекурсивные реализации с кешем и без для поиска расстояния Дамерау~---~Левенштейна имеют приблизительно одинаковое время работы, 
	    но с увеличением длины строки реализация без кеша выполняется на порядок дольше, поскольу не происходит повторное вычисление значений;
        \item разница между итеративными реализацими алгоритмов поиска расстояний Левенштейна и Дамерау~---~Левенштейна незначительна, и обусловлена она
        дополнительным условием на проверку равенства соседних символов для расстояния Дамерау~---~Левенштейна;
        \item итеративная реализация работает на порядок быстрее рекурсивной с кешем для поиска расстояния Дамерау~---~Левенштейна.
        \end{itemize}
    \item проведен сравнительный анализ затрачиваемой алгоритмами памяти: итеративные алгоритмы и рекурсивные алгоритмы с кешированием требуют больше памяти по сравнению с рекурсивным алгоритмом без кеширования. 
    В реализациях, использующих матрицы, максимальный используемый объем памяти увеличивается пропорционально произведению длин строк. С другой стороны, для рекурсивного алгоритма без кеширования потребление памяти 
    увеличивается пропорционально сумме длин строк.
\end{enumerate}