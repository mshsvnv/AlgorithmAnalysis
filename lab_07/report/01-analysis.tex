\chapter{Аналитический раздел}
В данном разделе будут рассмотрены алгоритм Кнута~--~Морриса~--~Пратта и его модификация.

Для этого введем следующие обозначения:
\begin{itemize}
	\item подстрока $x$ называется \textit{префиксом} строки $w$, если есть такая подстрока $z$, что $w = xz$;
	
	\item подстрока $x$ называется \textit{суффиксом} строки $w$, если есть такая подстрока $z$, что $w = zx$~\cite{lectures2}.
\end{itemize}

\section{Алгоритм Кнута~--~Морриса~--~Пратта}
Рассматриваемый алгоритм основывается на том, что после частичного совпадения начальной части подстроки с соответствующими символами строки фактически известна пройденная часть строки и можно, вычислить некоторые сведения, с помощью которых затем быстро продвинуться по строке.

Основным отличием алгоритма Кнута~--~Морриса~--~Пратта от алгоритма прямого поиска заключается в том, что сдвиг подстроки выполняется не на один символ на каждом шаге алгоритма, а на некоторое переменное количество символов. 
Следовательно, перед тем как осуществлять очередной сдвиг, необходимо определить величину сдвига~\cite{lectures2}.

Для определения этого сдвига используют \textit{префиксную функцию}. Это функция, которая для всякого префикса $b_1 \dots b_i$ строки $x = b_1 \dots b_m$ выдает длину наибольшего суффикса подстроки $b_1 \dots b_i$, который одновременно будет и префиксом $x$~\cite{lectures1}.

\section{Модификация алгоритма Кнута~--~Морриса~--~Пратта}
Модификация заключается в введении эвристики <<плохого>> символа по аналогии с алгоритмом Бойера~--~Мура с адаптацией к обходу подстроки слева направо.

Эвристика <<плохого>> символа определяет сдвиг наибольшего возможного количества позиций, если символ строки не совпадает с символом подстроки в заданной позиции~\cite{lectures1}.

Таким образом, в модифицированном алгоритме мы используем как информацию о суффиксах, так и о длине сдвига. 

\section*{Вывод}
В данном разделе были рассмотрены алгоритм Кнута~--~Морриса~--~Пратта и его модификация.