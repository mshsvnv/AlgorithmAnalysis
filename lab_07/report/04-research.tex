\chapter{Исследовательский раздел}
В данном разделе будут приведены примеры работы программ, постанов-
ка исследования и сравнительный анализ алгоритмов на основе полученных
данных.

\section{Технические характеристики}
Технические характеристики устройства, на котором выполнялись замеры по времени, представлены далее.
\begin{itemize}
	\item процессор: AMD Ryzen 5 5500U\,--\,2.10 ГГц;
	\item Оперативная память: 16 ГБайт;
	\item Операционная система: Windows 10 Pro 64-разрядная система версии 22H2.
\end{itemize}

При замерах времени ноутбук был включен в сеть электропитания и был нагружен только системными приложениями.

\section{Демонстрация работы программы}
На рисунке~\ref{img:prog_demo} представлена демонстрация работы разработанного ПО.  
\includeimage
{prog_demo}
{f}
{H}
{1\textwidth}
{Демонстрация работы программы}

\section{Число сравнений}
В качестве длины исходной строки были выбраны следующие значения: $256, 512, 1024, 2048, 4096$. Так же для получения более точного результата, каждый замер производился 100 раз.

Определим следующие случаи:
\begin{itemize}
	\item \textbf{лучший}: когда подстрока находится в начале строки;
	\item \textbf{худший}: когда подстрока находится в конце строки или отсутствует в целом.
\end{itemize}

На рисунке~\ref{img:fig_best} изображены результаты исследования для лучшего случая.
\includeimage
{fig_best}
{f}
{H}
{1\textwidth}
{Сравнение количества сравнений при работе алгоритмов для лучшего случая}

На рисунке~\ref{img:fig_worst1} изображены результаты исследования для худшего случая, когда искомая подстрока находится в конце строки.
\includeimage
{fig_worst1}
{f}
{H}
{1\textwidth}
{Сравнение количества сравнений при работе алгоритмов для лучшего случая (подстрока в конце строки)}

На рисунке~\ref{img:fig_worst2} изображены результаты исследования для худшего случая, когда искомой подстроки нет.
\includeimage
{fig_worst2}
{f}
{H}
{1\textwidth}
{Сравнение количества сравнений при работе алгоритмов для лучшего случая (подстроки нет)}

\section*{Вывод}
Наименьшее число сравнений получается в случае нахождения исходной подстроки в начале строки (для КМП и его модификации).

Количество сравнений в случае нахождения подстроки в начале исходной строки для алгоритма КМП и его модификации одинаково.

Самым худшим случаем является нахождение исходной подстроки в самом конце исходной строки, поскольку в таком случае алгоритм выполняет максимальное количество сравнений.


