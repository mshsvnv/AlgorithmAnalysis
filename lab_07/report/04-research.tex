\chapter{Исследовательский раздел}
В данном разделе будут приведены примеры работы программ, постанов-
ка исследования и сравнительный анализ алгоритмов на основе полученных
данных.

\section{Технические характеристики}
Технические характеристики устройства, на котором выполнялись замеры по времени, представлены далее.
\begin{itemize}
	\item Процессор: AMD Ryzen 5 5500U\,--\,2.10 ГГц;
	\item Оперативная память: 16 ГБайт;
	\item Операционная система: Windows 10 Pro 64-разрядная система версии 22H2.
\end{itemize}

При замерах времени ноутбук был включен в сеть электропитания и был нагружен только системными приложениями.

\section{Демонстрация работы программы}
На рисунке~\ref{} представлена демонстрация работы разработанного ПО.  

\includeimage
{prog_demo}
{f}
{H}
{1\textwidth}
{Демонстрация работы программы}

\section{Число сравнений}
В качестве длины исходной строки были выбраны следующие значения: $256, 512, 1024, 2048, 4096$. Так же для получения более точного результата, каждый замер производился 100 раз.

Определим следующие случаи:
\begin{itemize}
	\item \textbf{лучший}: когда подстрока находится в начале строки;
	\item \textbf{худший}: когда подстрока находится в конце строки или отсутствует в целом.
\end{itemize}

На рисунке~\ref{} изображены результаты исследования для лучшего случая.

На рисунке~\ref{} изображены результаты исследования для худшего случая, когда искомая подстрока находится в конце строки.

На рисунке~\ref{} изображены результаты исследования для худшего случая, когда искомой подстроки нет.

\section*{Вывод}