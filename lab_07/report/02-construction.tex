\chapter{Конструкторский раздел}
В данном разделе будут представлены псевдокоды алгоритмов Кнута~--~Морриса~--~Пратта и его модификации, а также алгоритмов для формирования массивов сдвигов и суффиксов.

\section{Требования к программному обеспечению}
К программному обеспечению предъявлен ряд требований:
\begin{enumerate}
	\item наличие интерфейса для выбора действия;
	\item возможность ввода строки и подстроки;
	\item возможность выполнения операции поиска подстроки в строке.
\end{enumerate}

\section{Разработка алгоритмов}
В качестве инcтрумента для создания псевдокода использован пакет \texttt{algorithm}.

В случае нахождения подстроки в строке просиходит возврат \textbf{shift}, то есть смещение относительно начала строки, иначе~--- \textbf{-1}. 

Определим следующие операторы и функции:
\begin{itemize}
	\item оператор $\gets$ обозначает присваивание значение переменной;
	\item оператор $[i]$ обозначает получение элемента из массива с индексом $i$;
	\item функция $\max$ возвращает максимальное значение;
	\item функция $ord$ возвращает целое число, представляющее символ Юникода;
	\item функция $length$ возвращает длину массива, строки.
\end{itemize}

В листингах~\ref{alg:alg_01}--\ref{alg:alg_02} рассмотрены псевдокоды алгоритмов формирования дополнительных массивов суффиксов и сдвигов соответственно. 

В листингах~\ref{alg:alg_03}--\ref{alg:alg_04} рассмотрены псевдокоды алгоритма Кнута~--~Морриса~--~Пратта и его модификации. 

\begin{algorithm}[H]
	\caption{Псевдокод алгоритма формирования массива суффиксов}
	\label{alg:alg_01}
	\text{\textbf{На входе}: строка $w = a_1 \dots a_n$, ссылка на массив $\pi$}\\
	\begin{algorithmic}[1]
		\Function{MakePiList}{$w$}
		\State $j \gets 0$
		\State $i \gets 1$
		\While{$i < n$}
			\If{$a_j = a_i$}
				\State $\pi[i] \gets j + 1$
				\State $j \gets j + 1$
				\State $i \gets i + 1$
			\Else
				\If{$j = 0$}
					\State $\pi[i] \gets 0$
					\State $i \gets i + 1$
				\Else
					\State $j \gets \pi[j - 1]$
				\EndIf
			\EndIf
		\EndWhile
		\EndFunction
	\end{algorithmic}
\end{algorithm}

\begin{algorithm}[H]
	\caption{Псевдокод алгоритма формирования массива сдвигов}
	\label{alg:alg_02}
	\text{\textbf{На входе}: строка $w = a_1 \dots a_n$, ссылка на массив $badChars$}\\
	\begin{algorithmic}[1]
		\Function{MakeBadChars}{$w$, $badChars$}
		\State $j \gets 0$
		\State $i \gets 1$
		\For{$i = (0, n)$}
			\State{$badChars[ord(w_i)] \gets i$}
		\EndFor
		\EndFunction
	\end{algorithmic}
\end{algorithm}

\begin{algorithm}[H]
	\caption{Псевдокод алгоритма  Кнута~--~Морриса~--~Пратта}
	\label{alg:alg_03}
	\text{\textbf{На входе}: строка $w = a_1 \dots a_n$, подстрока $x = b_1 \dots b_m$, массив суффиксов $\pi$}\\
	\begin{algorithmic}[1]
		\Function{KMP}{$w$, $x$, $\pi$}
		\State $i \gets 0$
		\State $j \gets 0$
		\State $n \gets length(w)$
		\State $m \gets length(x)$
		\While{$i < n$}
			\If{$a_i = b_j$}
				\State $i \gets i + 1$
				\State $j \gets j + 1$
				\If{$j = m$}
					\Return $i - n$
				\EndIf
			\Else
				\If{$j = 0$}
					\State $i \gets i + 1$
				\Else
					\State $j \gets \pi[j - 1]$
				\EndIf
			\EndIf
		\EndWhile
		
		\Return $-1$
		\EndFunction
	\end{algorithmic}
\end{algorithm}

\begin{algorithm}[H]
	\caption{Псевдокод модифицированного алгоритма  Кнута~--~Морриса~--~Пратта}
	\label{alg:alg_04}
	\text{\textbf{На входе}: строка $w = a_1 \dots a_n$, подстрока $x = b_1 \dots b_m$,}\\
	\text{массив суффиксов $\pi$, массив сдвигов $badChars$} \\
	\begin{algorithmic}[1]
		\Function{KMPopt}{$w$, $x$, $\pi$, $badChars$}
		\State $n \gets length(w)$
		\State $m \gets length(x)$
		\State $shift \gets 0$
		\While{$shift \leq n - m$}
			\State $j \gets m - 1$
			\While{$j \geq 0$ and $x_j = w_{shift + j}$}
				\State $j \gets m - 1$
			\EndWhile
			\If{$j < 0$}
				\Return $shift$
			\Else
				\State $shift \gets shift + \max(pi[j], badChars[ord(x_{shift + j})])$
			\EndIf
		\EndWhile
		
		\Return $-1$
		\EndFunction
	\end{algorithmic}
\end{algorithm}

\section*{Вывод}
В данном разделе были описаны псевдокоды для алгоритмов формирования массивов суффиксов и сдвигов, а также для алгоритма Кнута~--~Морриса~--~Пратта и его модификации. 