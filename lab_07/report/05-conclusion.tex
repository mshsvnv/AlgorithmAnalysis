\chapter*{\hfill{\centering  ЗАКЛЮЧЕНИЕ}\hfill}
\addcontentsline{toc}{chapter}{ЗАКЛЮЧЕНИЕ}

Цель данной лабораторной работы была достигнута: исследованы лучшие и худшие случаи работы алгоритмов поиска.

Для достижения поставленной цели были решены следующие задачи:
\begin{enumerate}
	\item описаны используемые алгоритмы поиска;
	\item выбраны средства программной реализации;
	\item реализованы алгоритмы поиска Кнута~--~Морриса~--~Пратта и его модификация;
	\item проанализированы алгоритмы по количеству сравнений и сделаны следующие выводы:
	\begin{itemize}
		\item наименьшее число сравнений получается в случае нахождения исходной подстроки в начале строки (для КМП и его модификации);
		\item количество сравнений в случае нахождения подстроки в начале исходной строки для алгоритма КМП и его модификации одинаково;
		\item самым худшим случаем является нахождение исходной подстроки в самом конце исходной строки, поскольку в таком случае алгоритм выполняет максимальное количество сравнений.
	\end{itemize}
\end{enumerate}
