\chapter*{Введение}
Задача поиска в строке заключается в следующем: дана длинная строка (<<текст>>) $w = a_1 \dots a_n$, и короткая искомая строка (<<шаблон>>) $x = b_1 \dots b_m$. Требуется найти все вхождения $x$ в $w$ в качестве подстроки, то есть, все смещения $s$, для которых подстрока $ws = a_{s + 1} \dots a_{s + m}$ совпадает с $b_1 \dots b_m$~\cite{lectures1}.

Целью данной лабораторной работы является исследование лучших и худших случаев работы алгоритмов поиска подстроки в строке: Кнута~--~Морриса~--~Пратта и его модификации.

Для достижения поставленной цели необходимо решить следующие задачи:
\begin{enumerate}
	\item описать используемые алгоритмы поиска;
	\item выбрать средства программной реализации;
	\item реализовать данные алгоритмы поиска;
	\item проанализировать алгоритмы по количеству сравнений.
\end{enumerate}
