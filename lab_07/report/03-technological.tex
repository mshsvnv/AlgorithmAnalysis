\chapter{Технологический раздел}
В данном разделе будут описаны средства реализации программного обеспечения, а также представлены листинги и функциональные тесты.

\section{Средства реализации}
Для реализации данной лабораторной работы был выбран язык \texttt{C++}, так как в нем есть стандартная библиотека \texttt{ctime}, которая позволяет производить замеры процессорного времени выполнения программы~\cite{cpp-lang}. 

\section{Сведения о модулях программы}
Данная программа разбита на следующие модули:
\begin{itemize}
	\item \texttt{main.cpp}~--- файл, содержащий точку входа в программу;
	\item \texttt{algs.cpp}~--- файл, содержащий реализации алгоритмов поиска.
\end{itemize}
	
\section{Реализация алгоритмов}
На листингах~\ref{lst:kmp.cpp}~---~\ref{lst:bad_chars.cpp} представлены реализации разрабатываемых алгоритмов.

\includelistingpretty
{kmp.cpp}
{c++}
{Реализация алгоритма Кнута~--~Морриса~--~Пратта поиска строки в подстроке}

\includelistingpretty
{kmp_modif.cpp}
{c++}
{Реализация модифицированного алгоритма Кнута~--~Морриса~--~Пратта поиска строки в подстроке}

\includelistingpretty
{suffixes.cpp}
{c++}
{Реализация алгоритма формирования массива суффиксов}

\includelistingpretty
{bad_chars.cpp}
{c++}
{Реализация алгоритма формирования массива сдвигов}

\section{Тестирование}
В таблице~\ref{t:mod_tests} приведены модульные тесты для разработанных алгоритмов поиска. Все тесты успешно пройдены.

\begin{table}[ht]
	\small
	\begin{center}
		\begin{threeparttable}
			\caption{Модульные тесты}
			\label{t:mod_tests}
			\begin{tabular}{|c|c|c|c|c|}
				\hline
				\bfseries Строка-текст
				& \bfseries Строка-образец
				& \bfseries Ожидаемый р-т
				& \multicolumn{2}{c|}{\bfseries Фактический р-т} \\ \cline{4-5}
				& & & \bfseries КМП & \bfseries Модиф. КМП \\
				\hline
				abcdefgabcdefg & bcd & 1 & 1 & 1 \\
				\hline
				abcdefghi & xyz & -1 & -1 & -1 \\
				\hline
				ababababab & abab & 0  & 0 & 0 \\
				\hline
				abcdefg & abcd & 0 & 0 & 0 \\
				\hline
				xyabc & abc & 2 & 2 & 2 \\
				\hline
			\end{tabular}	
		\end{threeparttable}	
	\end{center}
\end{table}

\section*{Вывод}
В данной части работы были представлены листинги реализованных алгоритмов и тесты, успешно пройденные программой.