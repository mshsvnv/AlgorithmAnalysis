\chapter{Технологический раздел}
В данном разделе будут описаны средства реализации программного обеспечения, а также представлены листинги и функциональные тесты.

\section{Средства реализации}
Для реализации данной лабораторной работы был выбран язык \texttt{C++}, так как в нем есть стандартная библиотека \texttt{ctime}, которая позволяет производить замеры процессорного времени выполнения программы~\cite{cpp-lang}. 

\section{Сведения о модулях программы}
Данная программа разбита на следующие модули:
\begin{itemize}
	\item \texttt{main.cpp}~--- файл, содержащий точку входа в программу;
	\item \texttt{algs.cpp}~--- файл, содержащий реализации алгоритмов поиска.
	\item \texttt{measure.cpp}~--- файл, содержащий функции, замеряющие процессорное время выполнения алгоритмов сортировок;
\end{itemize}
	
\section{Реализация алгоритмов}
На листингах~\ref{lst:kmp.cpp}~---~\ref{lst:bad_chars.cpp} представлены реализации разрабатываемых алгоритмов.

\includelistingpretty
{kmp.cpp}
{c++}
{Реализация алгоритма Кнута~--~Морриса~--~Пратта поиска строки в подстроке}

\includelistingpretty
{modif_kmp.cpp}
{c++}
{Реализация модифицированного алгоритма Кнута~--~Морриса~--~Пратта поиска строки в подстроке}

\includelistingpretty
{suffixes.cpp}
{c++}
{Реализация алгоритма формирования массива суффиксов}

\includelistingpretty
{bad_chars.cpp}
{c++}
{Реализация алгоритма формирования массива сдвигов}

\section{Тестирование}
В таблице~\ref{} приведены модульные тесты для разработанных алгоритмов поиска. Все тесты успешно пройдены.

\section*{Вывод}
В данной части работы были представлены листинги реализованных алгоритмов и тесты, успешно пройденные программой.