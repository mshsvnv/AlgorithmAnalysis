\chapter*{Введение}
\addcontentsline{toc}{chapter}{Введение}

\textbf{Расстояние Левенштейна} — это метрика, используемая для измерения разницы между двумя строками. Она вычисляет минимальное количество односимвольных изменений (вставок, удалений или замен), необходимых для преобразования одной строки в другую.

Расстояние Левенштейна обычно используется в различных областях, в том числе:
\begin{enumerate}
    \item \textbf{Проверка орфографии:} выявление и исправление ошибок для слов на основе их расстояния Левенштейна.
    \item \textbf{Анализ последовательности ДНК:} измерение сходства между последовательностями ДНК позволяет исследователям сравнивать и анализировать генетические данные.
    \item \textbf{Обработка естественного языка (НЛП):} используется в таких задачах, как классификация текстов, поиск информации и машинный перевод, для определения сходства между текстами.
\end{enumerate}

\textbf{Расстояние Дамерау-Левенштейна} является расширением расстояния Левенштейна, которое не учитывает транспозиции. Дополнительная операция позволяет обрабатывать случаи, когда символы меняются местами или переупорядочиваются.

\textbf{Целью} данной лабораторной работы является изучение, реализация и исследование алгоритмов поиска расстояний Левенштейна и Дамерау-Левенштейна.

Необходимо выполнить следующие \textbf{задачи}:
\begin{enumerate}[label={\arabic*)}]
    \item изучить алгоритмы Левенштейна и Дамерау-Левенштейна для нахождения редакционного расстояния между строками;
    \item реализовать данные алгоритмы;
    \item выполненить сравнительный анализ алгоритмов по затрачиваемым ресурсам (времени, памяти);
    \item описать и обосновать полученные результаты в отчете.
\end{enumerate}