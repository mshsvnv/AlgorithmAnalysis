\chapter*{Введение}
\addcontentsline{toc}{chapter}{Введение}

\textbf{Расстояние Левенштейна}~--- это метрика, используемая для измерения разницы между двумя строками. Она вычисляет минимальное количество односимвольных изменений (вставок, удалений или замен), необходимых для преобразования одной строки в другую.

Расстояние Левенштейна используется в различных областях~\cite{analysis-lev-damlev}:
\begin{enumerate}
    \item \textbf{проверка орфографии:} выявление и исправление ошибок для слов на основе их расстояния Левенштейна.
    \item \textbf{анализ последовательности ДНК:} измерение сходства между последовательностями ДНК позволяет исследователям сравнивать и анализировать генетические данные.
    \item \textbf{обработка естественного языка:} используется в таких задачах, как классификация текстов, поиск информации и машинный перевод, для определения сходства между текстами.
\end{enumerate}

\textbf{Расстояние Дамерау~---~Левенштейна}~--- это мера разницы двух строк, которая определяется наименьшим количеством необходимых действий (вставок, удалений, замен или перестановок соседних символов) для преобразования одной строки в другую.

\textbf{Целью} данной лабораторной работы является изучение, реализация и исследование алгоритмов поиска расстояний Левенштейна и Дамерау~---~Левенштейна.

Необходимо выполнить следующие \textbf{задачи}:
\begin{enumerate}[label={\arabic*)}]
    \item описать алгоритмы поиска расстояний Левенштейна и Дамерау~---~Левенштейна для нахождения редакционного расстояния между строками;
    \item реализовать данные алгоритмы;
    \item выполненить сравнительный анализ алгоритмов по затрачиваемым ресурсам (времени, памяти);
    \item описать и обосновать полученные результаты в отчете.
\end{enumerate}