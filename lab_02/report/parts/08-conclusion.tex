\chapter*{Заключение}
\addcontentsline{toc}{chapter}{Заключение}

В ходе исследования было определено, что оптимизированный алгоритм Винограда продемонстрировал лучшую производительность. 
В то же время стандартный метод умножения матриц работает медленнее по сравнению с обеими реализациями алгоритма Винограда. 
Внесенные модификации в оптимизированную реализацию также оказали влияние на ее скорость. 
Кроме того, при использовании алгоритма Винограда с четным размером матриц производительность выше, поскольку нет необходимости в дополнительных вычислениях для крайних строк и столбцов в случае нечетного размера. 
В результате рекомендуется использовать алгоритм Винограда при работе с матрицами четного размера.

Кроме того, реализация алгоритма Штрассена требует больше времени на выполнение из-за дополнительных операций сложения и вычитания матриц по сравнению с реализациями алгоритма Винограда.

С учетом теоретических расчетов потребляемой памяти можно сделать вывод о том, что стандартное умножение требует наименьшее количество памяти. 
Однако реализация алгоритма Штрассена требует наибольших затрат памяти из-за необходимости разбивать исходные матрицы на 4 подматрицы при каждом рекурсивном вызове, а также производить дополнительные операции сложения и умножения. 
Оптимизированная реализация алгоритма Винограда также требует больше памяти по сравнению с неоптимизированной из-за использования дополнительных переменных для хранения некоторых предварительных вычислений. 

Цель данной лабораторной работы была достигнута, а именно были исследованы алгоритмы умножения матриц.

В результате выполнения лабораторной работы для достижения этой цели были выполнены следующие задачи:
\begin{enumerate}
    \item описаны следующие алгоритмы умножения матриц:
        \begin{itemize}
            \item стандартный алгоритм умножения;
            \item алгоритм Винограда;
            \item оптимизированный алгоритм Винограда;
            \item алгоритм Штрассена.
        \end{itemize}
    \item релизованы описанные алгоритмы;
    \item дана оценка трудоемкости алгоритмов;
    \item дана оценка потребляемой памяти реализацими алгоритмов;
    \item проведены замеры времени выполнения алгоритмов;
\end{enumerate}

