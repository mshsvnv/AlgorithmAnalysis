\chapter{Технологическая часть}
В данном разделе будут приведены требования к программному обеспечению, средства реализации, листинг кода и функциональные тесты.

\section{Требования к ПО}
К программе предъявлен ряд требований:
\begin{itemize}
    \item входными данными являются две матрицы, каждя из которых хранится в файле, с расширением \texttt{*.txt};
    \item результат выполнения программы~--- матрица, полученная в результате умножения введенных матриц;
    \item возможность произвести замеры процессорного времени выполнения реализованных алгоритмов.
\end{itemize}

\section{Средства реализации}
Для реализации данной лабораторной работы был выбран язык \texttt{C++}~\cite{cpp-lang}, так как в нем есть стандартная библиотека \texttt{ctime}~\cite{cpp-lang}, которая позволяет производить замеры процессорного времени выполнения программы;

В качестве среды разработки был выбран \textit{Visual Studio Code}: он является кроссплатформенным и предоставляет полный набор интсрументов для проектирования и отладки кода.
 
\section{Сведения о модулях программы}
Данная программа разбита на следующие модули:

\begin{itemize}
	\item \texttt{main.cpp}~--- файл содержит точку входа в программу, из которой происходит вызов алгоритмов по разработанному интерфейсу;
	\item \texttt{matrix.cpp}~--- файл содержит реализацию класса \texttt{MatrixT};
	\item \texttt{multiply.cpp}~--- файл содержит функции умножения матриц;
	\item \texttt{measure.cpp}~--- файл содержит функции, замеряющие процессорное время выполнения алгоритмов умножения матриц;
\end{itemize}

\section{Реализация алгоритмов}
В листинге \ref{lst:std} приведена реализация алгоритма стандартного умножения матриц.

В листинге \ref{lst:vin} приведена реализация алгоритма умножения матриц методом Винограда.

В листинге \ref{lst:vin_opt} приведена реализация оптимизированного алгоритма умножения матриц методом Винограда.

В листинге \ref{lst:strassen} приведена реализация алгоритма умножения матриц методом Штрассена.

\lstinputlisting[label=lst:std, caption=Функция стандартного умножения матриц, firstline=16,lastline=36]{../code/src/multiply.cpp}

\clearpage

\lstinputlisting[label=lst:vin, caption=Функция умножения матриц методом Винограда, firstline=38,lastline=85]{../code/src/multiply.cpp}

\clearpage

\lstinputlisting[label=lst:vin_opt, caption=Оптимизированная функция умножения матриц методом Винограда, firstline=87,lastline=137]{../code/src/multiply.cpp}

\clearpage

\lstinputlisting[label=lst:strassen, caption=Функция умножения матриц методом Штрассена, firstline=139,lastline=192]{../code/src/multiply.cpp}

\clearpage


\section{Функциональные тесты}
В таблице \ref{tbl:func_test_std} приведены тесты для функции, реализующей алгоритм стандартного умножения матриц.

В таблице \ref{tbl:func_test_vin} приведены тесты для функции, реализующих алгоритмы умножения матриц методом Винограда.

В таблице \ref{tbl:func_test_strassen} приведены тесты для функции, реализующих алгоритмы умножения матриц методом Штрассена.

\begin{table}[h!]
    \caption{Функциональные тесты для стандартного алгоритма умножения матриц}
    \label{tbl:func_test_std}
    \centering
        \begin{tabular}{||c|c|c||} 
        \hline
        Матрица 1& Матрица 2& Ожидаемый результат \\
        \hline\hline
        $\begin{pmatrix}
            1 & 2\\
            3 & 4\\
        \end{pmatrix}$ 
        &  
        $\begin{pmatrix}
            &
        \end{pmatrix}$
        &
        \textit{Неверный размер} \\
        \hline
        $\begin{pmatrix}
            1 & 2\\
        \end{pmatrix}$ 
        &  
        $\begin{pmatrix}
            3 & 4\\
        \end{pmatrix}$
        &
        \textit{Неверный размер} \\
        \hline
        $\begin{pmatrix}
            1
        \end{pmatrix}$ 
        &  
        $\begin{pmatrix}
            1
        \end{pmatrix}$
        &
        $\begin{pmatrix}
            1
        \end{pmatrix}$ \\
        \hline
        $\begin{pmatrix}
            1 & 1 & 1\\
        \end{pmatrix}$ 
        &  
        $\begin{pmatrix}
            1\\
            1\\
            1\\
        \end{pmatrix}$
        &
        $\begin{pmatrix}
            3
        \end{pmatrix}$ \\
        \hline
        $\begin{pmatrix}
            5 & 6 & 7\\
            4 & 9 & 8\\
            3 & 2 & 1\\
        \end{pmatrix}$ 
        &  
        $\begin{pmatrix}
            1 & 0 & 0\\
            0 & 1 & 0\\
            0 & 0 & 1\\
        \end{pmatrix}$
        &
        $\begin{pmatrix}
            5 & 6 & 7\\
            4 & 9 & 8\\
            3 & 2 & 1\\
        \end{pmatrix}$ \\
        \hline
        $\begin{pmatrix}
            5 & 6 & 7\\
            4 & 9 & 8\\
            3 & 2 & 1\\
        \end{pmatrix}$ 
        &  
        $\begin{pmatrix}
            1 & 0\\
            0 & 1\\
            0 & 0\\
        \end{pmatrix}$
        &
        $\begin{pmatrix}
            5 & 6\\
            4 & 9\\
            3 & 2\\
        \end{pmatrix}$ \\
        \hline
        \end{tabular}
\end{table}

\begin{table}[h!]
    \caption{Функциональные тесты для алгоритма умножения матриц методом Винограда}
    \label{tbl:func_test_vin}
    \centering
        \begin{tabular}{||c|c|c||} 
        \hline
        Матрица 1& Матрица 2& Ожидаемый результат \\
        \hline\hline
        $\begin{pmatrix}
            1 & 2\\
            3 & 4\\
        \end{pmatrix}$ 
        &  
        $\begin{pmatrix}
            &
        \end{pmatrix}$
        &
        \textit{Неверный размер} \\
        \hline
        $\begin{pmatrix}
            1 & 2\\
        \end{pmatrix}$ 
        &  
        $\begin{pmatrix}
            3 & 4\\
        \end{pmatrix}$
        &
        \textit{Неверный размер} \\
        \hline
        $\begin{pmatrix}
            1
        \end{pmatrix}$ 
        &  
        $\begin{pmatrix}
            1
        \end{pmatrix}$
        &
        $\begin{pmatrix}
            1
        \end{pmatrix}$ \\
        \hline
        $\begin{pmatrix}
            1 & 1 & 1\\
        \end{pmatrix}$ 
        &  
        $\begin{pmatrix}
            1\\
            1\\
            1\\
        \end{pmatrix}$
        & \textit{Неверный размер} \\
        \hline
        $\begin{pmatrix}
            5 & 6 & 7\\
            4 & 9 & 8\\
            3 & 2 & 1\\
        \end{pmatrix}$ 
        &  
        $\begin{pmatrix}
            1 & 0 & 0\\
            0 & 1 & 0\\
            0 & 0 & 1\\
        \end{pmatrix}$
        &
        $\begin{pmatrix}
            5 & 6 & 7\\
            4 & 9 & 8\\
            3 & 2 & 1\\
        \end{pmatrix}$ \\
        \hline
        $\begin{pmatrix}
            5 & 6 & 7\\
            4 & 9 & 8\\
            3 & 2 & 1\\
        \end{pmatrix}$ 
        &  
        $\begin{pmatrix}
            1 & 0\\
            0 & 1\\
            0 & 0\\
        \end{pmatrix}$
        & \textit{Неверный размер} \\
        \hline
        \end{tabular}
\end{table}

\begin{table}[h!]
    \caption{Функциональные тесты для алгоритма умножения матриц методом Штрассена}
    \label{tbl:func_test_strassen}
    \centering
        \begin{tabular}{||c|c|c||} 
        \hline
        Матрица 1& Матрица 2& Ожидаемый результат \\
        \hline\hline
        $\begin{pmatrix}
            1 & 2\\
            3 & 4\\
        \end{pmatrix}$ 
        &  
        $\begin{pmatrix}
            &
        \end{pmatrix}$
        &
        \textit{Неверный размер} \\
        \hline
        $\begin{pmatrix}
            1 & 2\\
        \end{pmatrix}$ 
        &  
        $\begin{pmatrix}
            3 & 4\\
        \end{pmatrix}$
        &
        \textit{Неверный размер} \\
        \hline
        $\begin{pmatrix}
            1 & 1 & 1\\
        \end{pmatrix}$ 
        &  
        $\begin{pmatrix}
            1\\
            1\\
            1\\
        \end{pmatrix}$
        &
        \textit{Неверный размер} \\
        \hline
        $\begin{pmatrix}
            5 & 6 & 7\\
            4 & 9 & 8\\
            3 & 2 & 1\\
        \end{pmatrix}$ 
        &  
        $\begin{pmatrix}
            1 & 0 & 0\\
            0 & 1 & 0\\
            0 & 0 & 1\\
        \end{pmatrix}$
        &
        \textit{Неверный размер} \\
        \hline
        $\begin{pmatrix}
            1
        \end{pmatrix}$ 
        &  
        $\begin{pmatrix}
            1
        \end{pmatrix}$
        &
        $\begin{pmatrix}
            1
        \end{pmatrix}$ \\
        \hline
        $\begin{pmatrix}
            5 & 7\\
            4 & 8\\
        \end{pmatrix}$ 
        &  
        $\begin{pmatrix}
            1 & 0\\
            0 & 1\\
        \end{pmatrix}$
        &
        $\begin{pmatrix}
            5 & 7\\
            4 & 8\\
        \end{pmatrix}$ \\
        \hline
        $\begin{pmatrix}
            1 & 2 & 3 & 4\\
            5 & 6 & 7 & 8\\
            9 & 10 & 11 & 12\\
            13 & 14 & 14 & 16
        \end{pmatrix}$ 
        &  
        $\begin{pmatrix}
            1 & 0 & 0 & 0\\
            0 & 1 & 0 & 0\\
            0 & 0 & 1 & 0\\
            0 & 0 & 0 & 1
        \end{pmatrix}$
        &
        $\begin{pmatrix}
            1 & 2 & 3 & 4\\
            5 & 6 & 7 & 8\\
            9 & 10 & 11 & 12\\
            13 & 14 & 14 & 16
        \end{pmatrix}$ \\
        \hline
        \end{tabular}
\end{table}

\section*{Вывод}
Были представлены листинги функций, реализующих стандартный алгоритм умножения матриц, алгоритм Винограда, оптимизированный алгоритм Винограда и алгоритм Штрассена.
Также в данном разделе была представлена информация о выбранных средствах для разработки алгоритмов.