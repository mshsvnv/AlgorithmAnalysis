\chapter{Технологическая часть}
В данном разделе будут приведены требования к программному обеспечению, средства реализации, листинг кода и функциональные тесты.

\section{Требования к ПО}
К программе предъявлен ряд требований:
\begin{itemize}
    \item входными данными являются две матрицы, каждя из которых хранится в файле, с расширением \texttt{*.txt};
    \item результат выполнения программы~---матрица, полученная в результате умножения введенных матриц;
    \item возможность произвести замеры процессорного времени выполнения реализованных алгоритмов.
\end{itemize}

\section{Средства реализации}

Для реализации данной лабораторной работы был выбран язык \texttt{C++}~\cite{cpp-lang}, так как в нем есть стандартная библиотека \texttt{ctime}~\cite{cpp-lang-time}, которая позволяет производить замеры процессорного времени выполнения программы;

В качестве среды разработки был выбран \textit{Visual Studio Code}: он является кроссплатформенным и предоставляет полный функционал для проектирования и отладки кода.
 
\section{Сведения о модулях программы}

Данная программа разбита на следующие модули:

\begin{itemize}
	\item \texttt{main.cpp}~--- файл, содержащий точку входа в программу, из которой происходит вызов алгоритмов по разработанному интерфейсу;
	\item \texttt{algorithms.cpp}~--- файл содержит функции поиска расстояния Левенштейна и Дамерау~---~Левенштейна;
	\item \texttt{matrix.cpp}~--- файл содержит функции динамического выделения и очищения памяти для матрицы, а так же ее вывод на экран;
	\item \texttt{measure.cpp}~--- файл содержит функции, замеряющие процессорное время выполнения алгоритмов поиска расстояния Левенштейна и Дамерау~---~Левенштейна;
\end{itemize}

\section{Реализация алгоритмов}

В листингах \ref{}--\ref{} приведены реализации алгоритма стандартного умножения матриц.

В листингах \ref{}--\ref{} приведены реализации алгоритма умножения матриц методом Винограда.

В листингах \ref{}--\ref{} приведены реализации оптимизированного алгоритма умножения матриц методом Винограда.

\section{Функциональные тесты}

\begin{table}[ht]
	\small
	\begin{center}
		\begin{threeparttable}
		\caption{Функциональные тесты}
		\label{tbl:func_tests}
		\begin{tabular}{|c|c|c|c|c|c|}
			\hline
			\multicolumn{2}{|c|}{\bfseries Входные данные}
			& \multicolumn{4}{c|}{\bfseries Расстояние и алгоритм} \\ 
			\hline 
			&
			& \multicolumn{1}{c|}{\bfseries Левенштейна} 
			& \multicolumn{3}{c|}{\bfseries Дамерау~---~Левенштейна} \\ \cline{3-6}
			
			\bfseries Строка 1 & \bfseries Строка 2 & \bfseries Итеративный & \bfseries Итеративный
			
			& \multicolumn{2}{c|}{\bfseries Рекурсивный} \\ \cline{5-6}
			& & & & \bfseries Без кеша & \bfseries С кешем \\
			\hline
			a & b & 1 & 1 & 1 & 1 \\
			\hline
			a & a & 0 & 0 & 0 & 0 \\
			\hline
			кот & скат & 2 & 2 & 2 & 2 \\
			\hline
			кот & кто & 2 & 1 & 1 & 1 \\
			\hline
			Австралия & Австрия & 2 & 2 & 2 & 2 \\
			\hline
			кот & ток & 2 & 2 & 2 & 2 \\
			\hline
			слон & слоны & 1 & 1 & 1 & 1 \\
			\hline
		\end{tabular}	
		\end{threeparttable}
	\end{center}
\end{table}

\section*{Вывод}
Были реализованы и протестированы алгоритмы поиска расстояния Левенштейна итеративно, а также поиска расстояния Дамерау–Левенштейна итеративно, рекурсивно и рекурсивного с кеширования. Проведено тестирование реализаций алгоритмов.
