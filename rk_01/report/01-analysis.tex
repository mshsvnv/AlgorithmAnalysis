\chapter{Аналитический раздел}
В данном разделе приведена информация, касающаяся разреженных матриц и графовых моделей.

\section{Разреженный строчный формат матрицы}
Разреженный строчный формат (сокр. РСФ)~--- это одна из наиболее широко используемых схем хранения разреженных матриц~\cite{csr}.

Значения ненулевых элементов матрицы и соответствующие столбцовые индексы хранятся в этой схеме по строкам в двух массивах; назовем их соответственно $AN$ и $JA$. 
Используется также массив указателей (скажем, $IA$, еще обозначающийся как $NR$), отмечающих позиции массивов $AN$ и $JA$, с которых начинается описание очередной строки. Дополнительная компонента в $IA$ содержит указатель первой свободной позиции в $JA$ и $AN$. 

Рассмотрим матрицу $A$:
\begin{equation}
	\label{matrix}
	A = \begin{pmatrix}
		0 & 0 & 0 & 1. & 3. & 0 & 0 & 5. & 0 & 0 \\
		0 & 0 & 0 & 0 & 0 & 0 & 0 & 0 & 0 & 0 \\
		0 & 0 & 0 & 0 & 0 & 7. & 0 & 1. & 0 & 0 				
	\end{pmatrix}
\end{equation}

Разреженный строчный формат матрицы~\ref{matrix} представлен ниже:
\begin{table}[h!]
	\begin{center}
		\begin{tabular}{c c c c c c c c}
			позиция & = & 1 & 2 & 3 & 4 & 5 & 6 \\
			AN & = & 1. & 3. & 5. & 7. & 1. & \\
			JA & = & 3 & 4 & 8 & 6 & 8 & \\
			NR & = & 1 & 4 & 4 & 6 & & 
		\end{tabular}
	\end{center}
\end{table}

В общем случае описание i-й строки $А$ хранится в позициях с $IA(i)$ до $IA(i + 1)$~--- 1 массивов $JA$ и $AN$, за исключением равенства $IA(i + 1) = IA(i)$, означающего, что r-я строка пуста. Если матрица $А$ имеет $n$ строк, то $IA$ содержит $n + 1$ позиций. 
		
\section{Сложение РСФ матриц}
Для сложения матриц в РСФ необходимо пройти через матрицу указателей $IA$ поэлементно через две матрицы, выделяя интервалы между соседними элементами $IA$ по $AN$ и $JA$, то есть $IA(i)$ и $IA(i + 1)$.

Если встретится элементы с одинаковыми строками и столбцами, то производится сложение. Затем происходит повторные проходы по двум матрицам по отдельности для того чтобы добавить в результирующую матрицу остаточные значение~\cite{csr}.

\section*{Вывод}
В данном разделе было рассмотрено понятие разреженной матрицы, а также были описаны графовые модели и алгоритм сложения матриц.