\chapter*{ВВЕДЕНИЕ}
\addcontentsline{toc}{chapter}{ВВЕДЕНИЕ}

Целью данной лабораторной работы является параметризация метода решения задачи коммивояжера на основе муравьиного метода.

Для достижения поставленной цели, необходимо решить следующие задачи:
\begin{enumerate}
	\item описать задачу коммивояжера;
	\item описать методы решения задачи коммивояжера: метод полного перебора и метод на основе муравьиного алгоритма;
	\item реализовать данные алгоритмы;
	\item сравнить реализованные алгоритмы по времени выполнения;
	\item провести параметризацию муравьиного алгоритма.
\end{enumerate}

Выданный индивидуальный вариант для выполнения лабораторной работы:
\begin{itemize}
	\item неориентированный граф;
	\item с элитными муравьями;
	\item карта перемещения для воздухоплавателей;
	\item Гамильтонов цикл.
\end{itemize}
	