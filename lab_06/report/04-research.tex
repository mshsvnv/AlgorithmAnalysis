\chapter{Исследовательский раздел}

\section{Технические характеристики}
Технические характеристики устройства, на котором выполнялись замеры по времени, представлены далее.
\begin{itemize}
	\item процессор: AMD Ryzen 5 5500U\,--\,2.10 ГГц;
	\item Оперативная память: 16 ГБайт;
	\item Операционная система: Windows 10 Pro 64-разрядная система версии 22H2.
\end{itemize}

При замерах времени ноутбук был включен в сеть электропитания и был нагружен только системными приложениями.

\section{Демонстрация работы программы}

На рисунке~\ref{img:prog_demo} продемонстрирована работа программы для случая, когда пользователь выбрал пункт 2 <<Муравьиный алгоритм>>, выбрал датасет <<cities.csv>> и ввел следующие параметры:
\begin{itemize}
	\item коэффициент $\alpha$: $0.3$;
	\item коэффициент испарения: $0.2$;
	\item количество дней: $10$;
	\item количество элитных муравьев: $5$.
\end{itemize}

\includeimage
{prog_demo}
{f}
{H}
{.6\textwidth}
{Демонстрация работы программы}

\section{Временные характеристики}

Исследование реализаций алгоритмов по времени выполнения производилось при варьировании размера $N$ матриц смежности от 2 до 11 с шагом 1.

В силу того, что время работы реализаций алгоритмов может колебаться в связи с различными процессами, происходящими в системе, для обеспечения более точных результатов измерения для каждой реализации алгоритма повторялись 10 раз, а затем бралось их среднее арифметическое значение.

На рисунке~\ref{img:time} представлены результаты исследования реализаций алгоритмов по времени работы в зависимости от размера матриц смежности; таблица~\ref{tbl:time} содержит данные, по которым строились данные графики.

\includeimage
{time}
{f}
{H}
{1\textwidth}
{Сравнение реализаций алгоритмов по времени работы при изменении размера матрицы смежности}

\begin{table}[H]
	\centering
	\caption{Результаты измерений реализаций алгоритмов при изменении размера матрицы смежности}
	\label{tbl:time}
	\begin{tabular}{|r|r|r|}
		\hline
		\multicolumn{1}{|l|}{Размер матрицы} &
		\multicolumn{1}{l|}{Полный перебор} &
		\multicolumn{1}{l|}{Муравьиный алгоритм} \\ \hline
		2 &   0.000096 &   0.004733 \\ \hline
		3 &   0.000015 &   0.016570 \\ \hline
		4 &   0.000029 &   0.044153 \\ \hline
		5 &   0.000122 &   0.094063 \\ \hline
		6 &   0.000806 &   0.175447 \\ \hline
		7 &   0.006886 &   0.301783 \\ \hline
		8 &   0.070623 &   0.477132 \\ \hline
		9 &   0.673715 &   0.728651 \\ \hline
		10 &   7.379569 &   1.050827 \\ \hline
		11 &  88.714126 &   1.463882 \\ \hline
	\end{tabular}
\end{table}

\section{Постановка исследования}

Автоматическая параметризация была проведена на трех классах данных: \ref{par:class1}, \ref{par:class2} и \ref{par:class3}.
Алгоритм будет запущен для набора значений $\alpha, \rho \in (0, 1)$.

Итоговая таблица значений параметризации будет состоять из следующих колонок:
\begin{itemize}
	\item $\alpha$~--- коэффициент жадности;
	\item $\rho$~--- коэффициент испарения;
	\item \textit{days}~--- количество дней жизни колонии муравьёв;
	\item \textit{Result}~--- эталонный результат, полученный методом полного перебора для проведения данного исследования;
	\item \textit{Mistake}~--- разность полученного основанным на муравьином алгоритме методом значения и эталонного значения на данных значениях параметров, показатель качества решения.
\end{itemize}

Цель исследования~--- определить комбинацию параметров, которые позволяют решить задачу наилучшим образом для выбранного класса данных.
Качество решения зависит от количества дней и погрешности измерений.

\subsection{Класс данных 1}
\label{par:class1}

Класс данных 1 представляет собой матрицу смежности размером 10 элементов (небольшой разброс значений: от 1 до 2), которая представлена в~(\ref{eqn:kd1}).

\begin{equation}
	\label{eqn:kd1}
	K_{1} = \begin{pmatrix}
		0 & 1 & 1 & 2 & 2 & 1 & 1 & 1 & 2 \\ 
		1 & 0 & 1 & 2 & 1 & 1 & 2 & 1 & 1 \\ 
		1 & 1 & 0 & 2 & 2 & 1 & 1 & 2 & 2 \\ 
		2 & 2 & 2 & 0 & 1 & 2 & 1 & 2 & 2 \\ 
		2 & 1 & 2 & 1 & 0 & 2 & 2 & 1 & 1 \\ 
		1 & 1 & 1 & 2 & 2 & 0 & 1 & 1 & 2 \\ 
		1 & 2 & 1 & 1 & 2 & 1 & 0 & 2 & 2 \\ 
		1 & 1 & 2 & 2 & 1 & 1 & 2 & 0 & 2 \\ 
		2 & 1 & 2 & 2 & 1 & 2 & 2 & 2 & 0 
	\end{pmatrix}
\end{equation}

Для данного класса данных приведена таблица~\ref{tbl:cls1} с выборкой параметров, которые наилучшим образом решают поставленную задачу. Использованы следующие обозначения: 
\begin{itemize}
	\item Days~--- количество дней;
	\item Result~--- результат работы;
	\item Mistake~--- ошибка как отклонение решения от эталонного.
\end{itemize}

\begin{center}
	\captionsetup{justification=raggedright,singlelinecheck=off}
	\begin{longtable}[c]{|r|r|r|r|r|}
		\caption{Параметры для класса данных 1 \label{tbl:cls1}}\\ \hline
		\multicolumn{1}{|l|}{$\alpha$} &
		\multicolumn{1}{l|}{$\rho$} & 
		\multicolumn{1}{l|}{Days} & 
		\multicolumn{1}{l|}{Result} & 
		\multicolumn{1}{l|}{Mistake} \\ \hline
		0.1 & 0.3 &  10 &    9 &    0 \\
		0.1 & 0.3 &  50 &    9 &    0 \\
		0.1 & 0.3 & 100 &    9 &    0 \\
		0.1 & 0.3 & 300 &    9 &    0 \\
		0.1 & 0.3 & 500 &    9 &    0 \\ \hline
		0.1 & 0.4 &  10 &    9 &    0 \\
		0.1 & 0.4 &  50 &    9 &    0 \\
		0.1 & 0.4 & 100 &    9 &    0 \\
		0.1 & 0.4 & 300 &    9 &    0 \\
		0.1 & 0.4 & 500 &    9 &    0 \\ \hline
		0.1 & 0.7 &  10 &    9 &    0 \\
		0.1 & 0.7 &  50 &    9 &    0 \\
		0.1 & 0.7 & 100 &    9 &    0 \\
		0.1 & 0.7 & 300 &    9 &    0 \\
		0.1 & 0.7 & 500 &    9 &    0 \\ \hline
		0.2 & 0.5 &  10 &    9 &    0 \\
		0.2 & 0.5 &  50 &    9 &    0 \\
		0.2 & 0.5 & 100 &    9 &    0 \\
		0.2 & 0.5 & 300 &    9 &    0 \\
		0.2 & 0.5 & 500 &    9 &    0 \\ \hline
		0.2 & 0.7 &  10 &    9 &    0 \\
		0.2 & 0.7 &  50 &    9 &    0 \\
		0.2 & 0.7 & 100 &    9 &    0 \\
		0.2 & 0.7 & 300 &    9 &    0 \\
		0.2 & 0.7 & 500 &    9 &    0 \\ \hline
		0.3 & 0.4 &  10 &    9 &    0 \\
		0.3 & 0.4 &  50 &    9 &    0 \\
		0.3 & 0.4 & 100 &    9 &    0 \\
		0.3 & 0.4 & 300 &    9 &    0 \\
		0.3 & 0.4 & 500 &    9 &    0 \\ \hline
		0.3 & 0.5 &  10 &    9 &    0 \\
		0.3 & 0.5 & 100 &    9 &    0 \\
		0.3 & 0.5 & 300 &    9 &    0 \\
		0.3 & 0.5 & 500 &    9 &    0 \\ \hline
		0.4 & 0.5 &  10 &    9 &    0 \\
		0.4 & 0.5 &  50 &    9 &    0 \\
		0.4 & 0.5 & 100 &    9 &    0 \\
		0.4 & 0.5 & 300 &    9 &    0 \\
		0.4 & 0.5 & 500 &    9 &    0 \\ \hline
		0.6 & 0.1 &  10 &    9 &    0 \\
		0.6 & 0.1 &  50 &    9 &    0 \\
		0.6 & 0.1 & 100 &    9 &    0 \\
		0.6 & 0.1 & 300 &    9 &    0 \\
		0.6 & 0.1 & 500 &    9 &    0 \\ \hline
	\end{longtable}
\end{center}

\subsection{Класс данных 2}
\label{par:class2}

Класс данных 2 представляет собой матрицу смежности размером 9 элементов (большой разброс значений: от 1000 до 9999), которая представлена
далее.

\begin{equation}
	\label{eqт:kd2}
	K_{2} = \begin{pmatrix}
		0 & 9271 & 8511 & 2010 & 1983 & 7296 & 7289 & 3024 & 1011 \\
		9271 & 0 & 7731 & 4865 & 5494 & 6812 & 4755 & 7780 & 7641 \\
		8511 & 7731 & 0 & 1515 & 9297 & 7506 & 5781 & 5804 & 7334 \\
		2010 & 4865 & 1515 & 0 & 3662 & 9597 & 2876 & 8188 & 9227 \\
		1983 & 5494 & 9297 & 3662 & 0 & 8700 & 4754 & 7445 & 3834 \\
		7296 & 6812 & 7506 & 9597 & 8700 & 0 & 4216 & 5553 & 8215 \\
		7289 & 4755 & 5781 & 2876 & 4754 & 4216 & 0 & 4001 & 4715 \\
		3024 & 7780 & 5804 & 8188 & 7445 & 5553 & 4001 & 0 & 9522 \\
		1011 & 7641 & 7334 & 9227 & 3834 & 8215 & 4715 & 9522 & 0 
	\end{pmatrix}
\end{equation}

Для данного класса данных приведена таблица~\ref{tbl:cls2} с выборкой параметров, которые наилучшим образом решают поставленную задачу.

\begin{center}
	\captionsetup{justification=raggedright,singlelinecheck=off}
	\begin{longtable}[c]{|r|r|r|r|r|}
		\caption{Параметры для класса данных 2 \label{tbl:cls2}}\\ \hline
		\multicolumn{1}{|l|}{$\alpha$} &
		\multicolumn{1}{l|}{$\rho$} & 
		\multicolumn{1}{l|}{Days} & 
		\multicolumn{1}{l|}{Result} & 
		\multicolumn{1}{l|}{Mistake} \\ \hline
		0.1 &  0.3 &  100 & 34192 &     0 \\
		0.1 &  0.3 &  300 & 34192 &     0 \\
		0.1 &  0.3 &  500 & 34192 &     0 \\ \hline
		0.1 &  0.7 &  100 & 34192 &     0 \\
		0.1 &  0.7 &  300 & 34192 &     0 \\
		0.1 &  0.7 &  500 & 34192 &     0 \\ \hline
		0.2 &  0.1 &  100 & 34192 &     0 \\
		0.2 &  0.1 &  300 & 34192 &     0 \\
		0.2 &  0.1 &  500 & 34192 &     0 \\ \hline
		0.2 &  0.2 &  100 & 34192 &     0 \\
		0.2 &  0.2 &  300 & 34192 &     0 \\
		0.2 &  0.2 &  500 & 34192 &     0 \\ \hline
		0.2 &  0.5 &  100 & 34192 &     0 \\
		0.2 &  0.5 &  300 & 34192 &     0 \\
		0.2 &  0.5 &  500 & 34192 &     0 \\ \hline
		0.2 &  0.8 &  100 & 34192 &     0 \\
		0.2 &  0.8 &  300 & 34192 &     0 \\
		0.2 &  0.8 &  500 & 34192 &     0 \\ \hline
		0.3 &  0.1 &  100 & 34192 &     0 \\
		0.3 &  0.1 &  300 & 34192 &     0 \\
		0.3 &  0.1 &  500 & 34192 &     0 \\ \hline
		0.3 &  0.2 &    5 & 34192 &     0 \\
		0.3 &  0.2 &   50 & 34192 &     0 \\
		0.3 &  0.2 &  100 & 34192 &     0 \\
		0.3 &  0.2 &  300 & 34192 &     0 \\
		0.3 &  0.2 &  500 & 34192 &     0 \\ \hline
		0.4 &  0.5 &   50 & 34192 &     0 \\
		0.4 &  0.5 &  300 & 34192 &     0 \\
		0.4 &  0.5 &  500 & 34192 &     0 \\ \hline
		0.5 &  0.2 &  100 & 34192 &     0 \\
		0.5 &  0.2 &  300 & 34192 &     0 \\
		0.5 &  0.2 &  500 & 34192 &     0 \\ \hline
		0.6 &  0.2 &  100 & 34192 &     0 \\
		0.6 &  0.2 &  300 & 34192 &     0 \\
		0.6 &  0.2 &  500 & 34192 &     0 \\ \hline
		0.6 &  0.3 &  300 & 34192 &     0 \\
		0.6 &  0.3 &  500 & 34192 &     0 \\ \hline
		0.6 &  0.4 &  100 & 34192 &     0 \\
		0.6 &  0.4 &  500 & 34192 &     0 \\ \hline
		0.6 &  0.5 &  100 & 34192 &     0 \\
		0.6 &  0.5 &  300 & 34192 &     0 \\
		0.6 &  0.5 &  500 & 34192 &     0 \\ \hline
	\end{longtable}
\end{center}

\subsection{Класс данных 3}
\label{par:class3}

Класс данных 3 представляет собой матрицу расстояний между следующими городами: Москва, Великий Новгород, Тверь, Ярославль, Великий Устюг, Коломна, Нижний Новгород, Казань, Астрахань, Вологда~\cite{map}.

\begin{equation}
	\label{eqт:kd3}
	K_{3} = \begin{pmatrix}
		0 & 490 & 160 & 250 & 750 & 100 & 400 & 715 & 1270 & 410 \\
		490 & 0 & 330 & 515 & 880 & 590 & 800 & 1115 & 1760 & 700 \\
		160 & 330 & 0 & 255 & 740 & 265 & 500 & 820 & 1435 & 350 \\
		250 & 515 & 255 & 0 & 505 & 290 & 290 & 600 & 1370 & 180 \\
		750 & 880 & 740 & 505 & 0 & 770 & 510 & 575 & 1605 & 450 \\
		100 & 590 & 265 & 290 & 770 & 0 & 355 & 660 & 1170 & 465 \\
		400 & 800 & 500 & 290 & 510 & 355 & 0 & 325 & 1145 & 405 \\
		715 & 1115 & 820 & 600 & 575 & 660 & 325 & 0 & 1055 & 670 \\
		1270 & 1760 & 1435 & 1370 & 1605 & 1170 & 1145 & 1055 & 0 & 1530 \\
		410 & 700 & 350 & 180 & 450 & 465 & 405 & 670 & 1530 & 0 \\
	\end{pmatrix}
\end{equation}

Для данного класса данных приведена таблица~\ref{tbl:cls3} с выборкой параметров, которые наилучшим образом решают поставленную задачу.

\begin{center}
	\captionsetup{justification=raggedright,singlelinecheck=off}
	\begin{longtable}[c]{|r|r|r|r|r|}
		\caption{Параметры для класса данных 3 \label{tbl:cls3}}\\
		\hline
		\multicolumn{1}{|l|}{$\alpha$} &
		\multicolumn{1}{l|}{$\rho$} & 
		\multicolumn{1}{l|}{Days} & 
		\multicolumn{1}{l|}{Result} & 
		\multicolumn{1}{l|}{Mistake} \\ \hline
		0.1 &  0.3 &  100 & 3725 &     0 \\
		0.1 &  0.3 &  300 & 3725 &     0 \\
		0.1 &  0.3 &  500 & 3725 &     0 \\ \hline
		0.1 &  0.7 &  100 & 3725 &     0 \\
		0.1 &  0.7 &  300 & 3725 &     0 \\
		0.1 &  0.7 &  500 & 3725 &     0 \\ \hline
		0.2 &  0.1 &  100 & 3725 &     0 \\
		0.2 &  0.1 &  300 & 3725 &     0 \\
		0.2 &  0.1 &  500 & 3725 &     0 \\ \hline
		0.2 &  0.2 &  100 & 3725 &     0 \\
		0.2 &  0.2 &  300 & 3725 &     0 \\
		0.2 &  0.2 &  500 & 3725 &     0 \\ \hline
		0.2 &  0.5 &  100 & 3725 &     0 \\
		0.2 &  0.5 &  300 & 3725 &     0 \\
		0.2 &  0.5 &  500 & 3725 &     0 \\ \hline
		0.2 &  0.8 &  100 & 3725 &     0 \\
		0.2 &  0.8 &  300 & 3725 &     0 \\
		0.2 &  0.8 &  500 & 3725 &     0 \\ \hline
		0.3 &  0.1 &  100 & 3725 &     0 \\
		0.3 &  0.1 &  300 & 3725 &     0 \\
		0.3 &  0.1 &  500 & 3725 &     0 \\ \hline
		0.3 &  0.2 &    5 & 3725 &     0 \\
		0.3 &  0.2 &   50 & 3725 &     0 \\
		0.3 &  0.2 &  100 & 3725 &     0 \\
		0.3 &  0.2 &  300 & 3725 &     0 \\
		0.3 &  0.2 &  500 & 3725 &     0 \\ \hline
		0.4 &  0.5 &   50 & 3725 &     0 \\
		0.4 &  0.5 &  300 & 3725 &     0 \\
		0.4 &  0.5 &  500 & 3725 &     0 \\ \hline
		0.5 &  0.2 &  100 & 3725 &     0 \\
		0.5 &  0.2 &  300 & 3725 &     0 \\
		0.5 &  0.2 &  500 & 3725 &     0 \\ \hline
		0.6 &  0.2 &  100 & 3725 &     0 \\
		0.6 &  0.2 &  300 & 3725 &     0 \\
		0.6 &  0.2 &  500 & 3725 &     0 \\ \hline
		0.6 &  0.3 &  300 & 3725 &     0 \\
		0.6 &  0.3 &  500 & 3725 &     0 \\ \hline
		0.6 &  0.4 &  100 & 3725 &     0 \\
		0.6 &  0.4 &  500 & 3725 &     0 \\ \hline
		0.6 &  0.5 &  100 & 3725 &     0 \\
		0.6 &  0.5 &  300 & 3725 &     0 \\
		0.6 &  0.5 &  500 & 3725 &     0 \\ \hline
	\end{longtable}
\end{center}

\section{Вывод}

В результате исследования было получено, что использование муравьиного алгоритма наиболее эффективно при больших размерах матриц.
Так, при размере матрицы, равном 2, муравьиный алгоритм медленнее алгоритма полного перебора в 49 раз, а при размере матрицы, равном 10, муравьиный алгоритм быстрее алгоритма полного перебора в 7 раз, а при размере в 11~--- уже в 60.5 раз.
Следовательно, при размерах матриц больше 9 следует использовать муравьиный алгоритм, но стоит учитывать, что он не гарантирует получения глобального оптимума при решении задачи.

Для класса данных 3 (см. п. \ref{par:class3}) было получено, что наилучшим образом алгоритм работает на значениях параметров, которые представлены далее:
\begin{itemize}[label=---]
	\item $\alpha = 0.1, \rho = 0.3, 0.7$;
	\item $\alpha = 0.2, \rho = 0.1, 0.2, 0.5, 0.8$;
	\item $\alpha = 0.3, \rho = 0.1, 0.2$;
	\item $\alpha = 0.4, \rho = 0.5$;
	\item $\alpha = 0.5, \rho = 0.2$;
	\item $\alpha = 0.6, \rho = 0.2, 0.3, 0.4$.
\end{itemize} 
Для этого класса данных рекомендуется использовать данные параметры.


